
\subsection{CNN}

CNN là kiến trúc mạng neural chuyên xử lý dữ liệu dạng lưới như hình ảnh.
\subsubsection{Kiến trúc cơ bản}

\textbf{1. Convolutional Layer:}
\begin{equation}
Y_{i,j} = \sigma\left(\sum_{m}\sum_{n} W_{m,n} \cdot X_{i+m,j+n} + b\right)
\end{equation}

trong đó $X$ là ảnh đầu vào, $W$ là kernel (bộ lọc), $b$ là bias, 
$\sigma(\cdot)$ là hàm kích hoạt (ReLU), và $Y_{i,j}$ là giá trị đặc trưng tại vị trí $(i,j)$.

\textbf{2. Pooling Layer:}
\begin{equation}
Y_{i,j} = \max_{m,n \in R} X_{i+m,j+n}
\end{equation}

trong đó $R$ là vùng pooling (thường kích thước $2 \times 2$), 
và $Y_{i,j}$ là giá trị sau khi giảm kích thước không gian.

\textbf{3. Fully Connected Layer:}
\begin{equation}
y = \sigma(W \cdot x + b)
\end{equation}

trong đó $x$ là vector đặc trưng đầu vào, $W$ là ma trận trọng số, 
$b$ là bias và $y$ là đầu ra của lớp kết nối đầy đủ.

\textbf{4. Softmax (Output):}
\begin{equation}
P(y=k) = \frac{e^{z_k}}{\sum_{j=1}^{K} e^{z_j}}
\end{equation}

trong đó $z_k$ là giá trị logit của lớp $k$, $K$ là tổng số lớp,
và $P(y=k)$ là xác suất dự đoán mẫu thuộc về lớp $k$.


\subsubsection{Ứng dụng trong hệ thống}

\textbf{Nhiệm vụ:} Nhận dạng ký tự trên biển số xe

\textbf{Kiến trúc mô hình:}
\begin{center}
\begin{tabular}{l}
Đầu vào (32×32×1) \\
$\rightarrow$ Conv2D(32, 3×3) $\rightarrow$ ReLU $\rightarrow$ MaxPool(2×2) \\
$\rightarrow$ Conv2D(64, 3×3) $\rightarrow$ ReLU $\rightarrow$ MaxPool(2×2) \\
$\rightarrow$ Flatten $\rightarrow$ Dense(128) $\rightarrow$ Dropout(0.5) \\
$\rightarrow$ Dense(36) $\rightarrow$ Softmax
\end{tabular}
\end{center}

\textbf{Đầu ra:} 36 lớp ký tự (A–Z và 0–9)

\textbf{Quá trình huấn luyện:}
\begin{itemize}
    \item Hàm mất mát: Categorical Cross-Entropy
    \item Thuật toán tối ưu: Adam (learning rate = 0.001)
    \item Tập dữ liệu: hơn 100.000 ảnh ký tự, có áp dụng tăng cường dữ liệu (data augmentation)
\end{itemize}
