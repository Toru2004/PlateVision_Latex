\subsection{Các công nghệ và kiến trúc hệ thống đề xuất}

Hệ thống quản lý bãi đỗ xe thông minh được xây dựng dựa trên sự tích hợp
giữa các công nghệ Internet of Things (IoT), trí tuệ nhân tạo (AI) và các
nền tảng ứng dụng hiện đại, nhằm đảm bảo khả năng nhận dạng chính xác,
xử lý thời gian thực và vận hành ổn định trong môi trường thực tế.

\subsubsection{Vi điều khiển ESP32 và chức năng IoT trong hệ thống bãi đỗ xe}

ESP32 được sử dụng làm vi điều khiển trung tâm trong mô hình hệ thống IoT
mô phỏng, nhằm minh họa nguyên lý kết nối, điều khiển và giám sát thiết bị
trong bãi đỗ xe thông minh. Việc sử dụng ESP32 giúp hệ thống dễ triển khai
và phù hợp cho mục đích nghiên cứu.

Chức năng IoT trong mô hình hệ thống:
\begin{itemize}
    \item Mô phỏng việc nhận lệnh điều khiển theo thời gian thực từ hệ thống trung tâm;
    \item Mô phỏng điều khiển cổng ra vào thông qua servo motor;
    \item Mô phỏng việc thu thập dữ liệu từ các cảm biến để phát hiện xe và vật cản;
    \item Mô phỏng cơ chế cảnh báo khi phát hiện tác động bất thường lên cổng;
    \item Gửi trạng thái hoạt động của hệ thống về server để phục vụ giám sát.
\end{itemize}



\subsubsection{Ứng dụng di động với Kotlin Jetpack Compose}

Ứng dụng Android được phát triển bằng Kotlin và Jetpack Compose theo mô hình
UI khai báo (declarative), giúp đơn giản hóa việc quản lý giao diện và trạng thái.
Các chức năng chính:
\begin{itemize}
    \item Đăng nhập/đăng ký người dùng;
    \item Đăng ký thông tin cá nhân phương tiện;
    \item Thực hiện check-in/check-out;
    \item Xem lịch sử ra vào (bảo vệ) và nhận thông báo thời gian thực.
\end{itemize}

\subsubsection{Hệ thống web quản trị với Nuxt.js}
Hệ thống web quản trị được xây dựng trên nền tảng Nuxt.js, cung cấp giao diện
quản lý bãi đỗ xe trực quan và hiệu quả cho quản trị viên.
Chức năng chính của dashboard:
\begin{itemize}
    \item Theo dõi số lượng xe và tình trạng bãi đỗ;
    \item Quản lý phương tiện và người dùng;
    \item Xuất báo cáo thống kê;
    \item Cập nhật dữ liệu theo thời gian thực.
\end{itemize}

\subsubsection{Luồng xử lý và logic quyết định}

Quyết định mở cổng được đưa ra dựa trên kết quả nhận dạng và xác thực:

\begin{equation}
\text{OPEN\_GATE} =
\begin{cases}
\text{True}, & \text{if } (plate\_match \land auth\_match) \\
\text{False}, & \text{otherwise}
\end{cases}
\end{equation}

Trong đó:
\begin{itemize}
    \item $\text{OPEN\_GATE}$: Trạng thái điều khiển cổng (mở hoặc đóng);
    \item $\text{plate\_match}$: Kết quả so khớp biển số xe với cơ sở dữ liệu
    (True nếu trùng khớp);
    \item $\text{auth\_match}$: Kết quả xác thực người dùng hoặc phương tiện
    (ví dụ: khuôn mặt, phương tiện đã đăng ký);
    \item $\land$: Phép toán logic \textit{AND}.
\end{itemize}



Luồng xử lý này đảm bảo hệ thống vận hành chính xác, an ninh và phù hợp với
mô hình bãi đỗ xe thông minh trong thực tế.
