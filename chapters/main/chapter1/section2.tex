\section{Các mô hình học sâu trong thị giác máy tính}

\subsection{CNN và YOLO}

Thị giác máy tính (Computer Vision) là một lĩnh vực quan trọng trong trí tuệ nhân tạo, tập trung vào việc giúp máy tính có khả năng hiểu và phân tích thông tin từ hình ảnh hoặc video. Trong đó, phát hiện đối tượng (object detection) là bài toán nhằm xác định vị trí và phân loại các đối tượng xuất hiện trong ảnh.

\textbf{Convolutional Neural Network (CNN)} là một mô hình học sâu được thiết kế đặc biệt cho dữ liệu hình ảnh. CNN sử dụng các lớp tích chập (convolutional layers) để tự động trích xuất đặc trưng từ ảnh đầu vào, giúp mô hình nhận biết các đặc điểm như cạnh, hình dạng và cấu trúc của đối tượng. Nhờ khả năng học đặc trưng hiệu quả, CNN đã trở thành nền tảng cho nhiều bài toán trong thị giác máy tính như phân loại ảnh, phát hiện đối tượng và nhận dạng khuôn mặt.

Dựa trên nền tảng CNN, \textbf{YOLO (You Only Look Once)} là một mô hình phát hiện đối tượng theo thời gian thực, cho phép xác định đồng thời vị trí và nhãn của các đối tượng chỉ trong một lần suy luận. YOLO chia ảnh đầu vào thành các lưới và dự đoán bounding box cùng xác suất lớp cho từng đối tượng. Ưu điểm nổi bật của YOLO là tốc độ xử lý nhanh, độ chính xác cao và khả năng triển khai hiệu quả trong các hệ thống yêu cầu xử lý thời gian thực.

Trong phạm vi đề tài, mô hình YOLO được sử dụng để phát hiện các đối tượng cần quan tâm trong hình ảnh hoặc video, làm cơ sở cho các bước xử lý và nhận dạng tiếp theo trong hệ thống.

\subsection{Siamese Network và DeepFace}

Bên cạnh bài toán phát hiện, nhận dạng (recognition) là bước nâng cao nhằm xác định danh tính hoặc mức độ tương đồng giữa các đối tượng đã được phát hiện. Bài toán này thường được áp dụng trong các hệ thống nhận dạng khuôn mặt hoặc so sánh các đối tượng có hình dạng tương tự nhau, chẳng hạn như phương tiện giao thông.

\textbf{Siamese Network} là một kiến trúc mạng nơ-ron đặc biệt gồm hai hoặc nhiều nhánh mạng có chung trọng số. Mô hình này không trực tiếp phân loại đối tượng mà học cách đo lường mức độ tương đồng giữa hai đầu vào thông qua không gian đặc trưng. Siamese Network thường được sử dụng trong các bài toán so sánh cặp ảnh, nhận dạng khuôn mặt, chữ viết tay hoặc nhận dạng các đối tượng có hình dạng tương tự nhau.

\textbf{DeepFace} là một mô hình học sâu chuyên biệt cho bài toán nhận dạng khuôn mặt, sử dụng mạng nơ-ron sâu để trích xuất đặc trưng khuôn mặt và biểu diễn chúng dưới dạng vector đặc trưng. Thông qua việc so sánh khoảng cách giữa các vector này, hệ thống có thể xác định hai khuôn mặt có thuộc cùng một người hay không. DeepFace cho phép nhận dạng khuôn mặt với độ chính xác cao ngay cả khi có sự thay đổi về góc nhìn, ánh sáng hoặc biểu cảm.

Trong đề tài, Siamese Network và DeepFace được áp dụng cho các bài toán nhận dạng như nhận dạng khuôn mặt và so sánh các đối tượng có đặc điểm tương đồng (ví dụ: xe có hình dạng giống nhau). Sự kết hợp giữa các mô hình phát hiện và nhận dạng giúp hệ thống hoạt động chính xác, linh hoạt và phù hợp với các kịch bản thực tế.

