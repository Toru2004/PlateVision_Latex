\section{Công nghệ và thiết bị sử dụng}

Trong đề tài “Bãi đỗ xe thông minh tích hợp AI”, nhóm sử dụng nhiều công nghệ hiện đại để đảm bảo tính chính xác, ổn định và khả năng mở rộng của hệ thống. Ở phía xử lý hình ảnh và trí tuệ nhân tạo, ngôn ngữ Python được kết hợp với thư viện OpenCV để xử lý dữ liệu video từ camera, mô hình YOLO cho bài toán phát hiện biển số và đối tượng, cùng với CNN và mô hình Siamese để thực hiện nhận diện khuôn mặt, so sánh hình ảnh đầu – đuôi xe và kiểm tra logo xe. Dữ liệu được lưu trữ và đồng bộ thông qua Firebase và Cloudinary nhằm đảm bảo truy cập nhanh, tiện lợi và an toàn.

Ở phía phần mềm ứng dụng, nhóm phát triển ứng dụng mobile bằng Kotlin để phục vụ người dùng và bảo vệ, đồng thời xây dựng hệ thống web quản trị bằng NuxtJS và Tailwind CSS cho giao diện, kết hợp với NestJS ở tầng backend để quản lý dữ liệu và xử lý logic nghiệp vụ. Về phần IoT, hệ thống sử dụng ESP32 và Arduino để điều khiển cổng tự động, kết nối với các thiết bị cảnh báo, hiển thị và barrier ra vào. Hệ thống thanh toán được tích hợp qua VNPAY nhằm hỗ trợ giao dịch trực tuyến nhanh chóng, tiện lợi và an toàn. Cuối cùng, sản phẩm được triển khai trên nền tảng Render giúp dễ dàng mở rộng và đảm bảo tính sẵn sàng cao khi vận hành thực tế.