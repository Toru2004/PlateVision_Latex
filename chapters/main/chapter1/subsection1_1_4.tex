\subsection{Mạng Siamese Network}

Siamese Network là kiến trúc mạng nơ-ron gồm hai nhánh CNN chia sẻ trọng số, 
được thiết kế để học độ tương đồng giữa hai ảnh đầu vào thông qua khoảng cách 
trong không gian đặc trưng.

Hai ảnh đầu vào được đưa qua mạng CNN để trích xuất embedding $\mathbf{f}_1$ 
và $\mathbf{f}_2$. Khoảng cách Euclid giữa hai embedding được tính:
\begin{equation}
d(\mathbf{f}_1, \mathbf{f}_2) = \|\mathbf{f}_1 - \mathbf{f}_2\|_2
\end{equation}

Hàm mất mát Contrastive Loss được sử dụng để huấn luyện:
\begin{equation}
\mathcal{L}_{contrastive} = y \, d^2 + (1-y)\max(0, m - d)^2
\end{equation}

trong đó $y \in \{0,1\}$ biểu thị hai mẫu có cùng lớp hay không, 
$d$ là khoảng cách embedding, và $m$ là margin phân tách các cặp khác lớp.
\textbf{Ứng dụng vào hệ thống bãi đỗ xe thông minh:}
\begin{itemize}
    \item Mạng Siamese Network được huấn luyện trên tập dữ liệu VeRi-776, bao gồm khoảng 32.000 ảnh đầu xe, nhằm phục vụ việc so sánh và xác thực phương tiện khi ra vào bãi đỗ.
    
    \item Khi xe đi vào bãi, hệ thống camera chụp ảnh đầu xe và trích xuất vector đặc trưng (embedding) $\mathbf{f}_{in}$; embedding này được lưu trữ trong cơ sở dữ liệu.
    
    \item Khi xe rời khỏi bãi, hệ thống tiếp tục trích xuất embedding $\mathbf{f}_{out}$ từ ảnh đầu xe tại cổng ra và tiến hành so sánh với embedding $\mathbf{f}_{in}$ đã lưu.
    
    \item Nếu khoảng cách $d(\mathbf{f}_{in}, \mathbf{f}_{out}) < \tau$ (với $\tau$ là ngưỡng xác định trước), phương tiện được xác nhận là cùng một xe.
    
    \item Ngược lại, nếu khoảng cách lớn hơn hoặc bằng ngưỡng $\tau$, hệ thống sẽ phát hiện và cảnh báo hành vi bất thường.
\end{itemize}
