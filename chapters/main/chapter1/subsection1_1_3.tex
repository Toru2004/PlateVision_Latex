\subsection{Thuật toán YOLO (You Only Look Once)}

YOLO là thuật toán object detection real-time, xử lý toàn bộ ảnh trong một lần forward pass.

\subsubsection{Nguyên lý hoạt động}

\textbf{1. Grid Division:}
Chia ảnh thành lưới $S \times S$ (ví dụ: 13×13)

\textbf{2. Bounding Box Prediction:}
Mỗi grid cell dự đoán $B$ bounding boxes, mỗi box có:
\begin{equation}
\text{Box} = (x, y, w, h, \text{confidence})
\end{equation}

\textbf{3. Confidence Score:}
\begin{equation}
\text{Confidence} = P(\text{Object}) \times \text{IOU}_{\text{pred}}^{\text{truth}}
\end{equation}

\textbf{4. Class Probability:}
\begin{equation}
P(\text{Class}_i | \text{Object}) \times \text{Confidence} = P(\text{Class}_i) \times \text{IOU}_{\text{pred}}^{\text{truth}}
\end{equation}

\textbf{5. Loss Function:}
\begin{equation}
\begin{split}
L = \lambda_{\text{coord}} \sum_{i=0}^{S^2} \sum_{j=0}^{B} \mathbb{1}_{ij}^{\text{obj}} [(x_i - \hat{x}_i)^2 + (y_i - \hat{y}_i)^2] \\
+ \lambda_{\text{coord}} \sum_{i=0}^{S^2} \sum_{j=0}^{B} \mathbb{1}_{ij}^{\text{obj}} [(\sqrt{w_i} - \sqrt{\hat{w}_i})^2 + (\sqrt{h_i} - \sqrt{\hat{h}_i})^2] \\
+ \sum_{i=0}^{S^2} \sum_{j=0}^{B} \mathbb{1}_{ij}^{\text{obj}} (C_i - \hat{C}_i)^2 \\
+ \lambda_{\text{noobj}} \sum_{i=0}^{S^2} \sum_{j=0}^{B} \mathbb{1}_{ij}^{\text{noobj}} (C_i - \hat{C}_i)^2 \\
+ \sum_{i=0}^{S^2} \mathbb{1}_{i}^{\text{obj}} \sum_{c \in \text{classes}} (p_i(c) - \hat{p}_i(c))^2
\end{split}
\end{equation}
trong đó:
\begin{itemize}
    \item $S$ là số ô lưới theo mỗi chiều của ảnh đầu vào;
    \item $B$ là số bounding box được dự đoán tại mỗi ô lưới;
    \item $(x_i, y_i, w_i, h_i)$ và $(\hat{x}_i, \hat{y}_i, \hat{w}_i, \hat{h}_i)$
    lần lượt là tọa độ dự đoán và tọa độ ground truth của bounding box;
    \item $C_i$ và $\hat{C}_i$ là độ tin cậy (confidence) dự đoán và giá trị thực;
    \item $p_i(c)$ và $\hat{p}_i(c)$ là xác suất dự đoán và nhãn thực của lớp $c$;
    \item $\mathbb{1}_{ij}^{\text{obj}}$ là hàm chỉ thị, bằng 1 nếu bounding box $j$
    tại ô lưới $i$ chứa đối tượng, ngược lại bằng 0;
    \item $\mathbb{1}_{ij}^{\text{noobj}}$ là hàm chỉ thị cho các ô không chứa đối tượng;
    \item $\lambda_{\text{coord}}$ và $\lambda_{\text{noobj}}$ là các hệ số cân bằng
    giữa sai số định vị và sai số độ tin cậy trong quá trình huấn luyện.
\end{itemize}

\subsubsection{Ứng dụng trong hệ thống}

Trong hệ thống đề xuất, ba mô hình YOLO được triển khai nhằm phục vụ các
nhiệm vụ nhận dạng khác nhau, bao gồm phát hiện biển số xe, nhận dạng ký tự
trên biển số và phát hiện logo phương tiện.

\textbf{1. Mô hình YOLO 1 – Phát hiện biển số xe:}
\begin{itemize}
    \item Đầu vào: Ảnh phương tiện đầy đủ, kích thước $640 \times 640$, tập dữ liệu gồm khoảng 4.500 ảnh;
    \item Số lớp: 1 lớp (license\_plate);
    \item Đầu ra: Tọa độ bounding box của biển số xe trong ảnh.
\end{itemize}

\textbf{2. Mô hình YOLO 2 – Phát hiện và phân tách ký tự:}
\begin{itemize}
    \item Đầu vào: Ảnh biển số đã được cắt (crop), kích thước $416 \times 416$, tập dữ liệu gồm khoảng 2.000 ảnh;
    \item Số lớp: 1 lớp (character\_area);
    \item Đầu ra: Vị trí và nhãn của từng ký tự trên biển số.
\end{itemize}

\textbf{3. Mô hình YOLO 3 – Phát hiện logo hãng xe:}
\begin{itemize}
    \item Đầu vào: Ảnh phần đầu xe ô tô, kích thước $640 \times 640$, tập dữ liệu gồm khoảng 18.000 ảnh;
    \item Số lớp: 20 lớp logo thương hiệu (Toyota, Honda, Mazda, \ldots);
    \item Đầu ra: Bounding box và nhãn thương hiệu logo tương ứng.
\end{itemize}

\textbf{Cấu hình huấn luyện và triển khai:}
\begin{itemize}
    \item Backbone: CSPDarknet53;
    \item Phiên bản mô hình: YOLOv8;
    \item Tốc độ xử lý: từ 60 đến 140 FPS khi triển khai trên GPU;
    \item Độ chính xác: mAP@0.5 đạt trên 90\%.
\end{itemize}
