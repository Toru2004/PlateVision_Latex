\section{Cơ sở khoa học và ý nghĩa thực tiễn của đề tài}

\subsection{Cơ sở khoa học}
Đề tài ``Nghiên cứu và ứng dụng Deep Learning kết hợp IoT trong xây dựng hệ thống bãi đỗ xe thông minh cho Trường Đại học Giao thông Vận tải TP. Hồ Chí Minh'' dựa trên các nền tảng khoa học và kỹ thuật sau:

\subsubsection{Trí tuệ nhân tạo và học sâu}
Hệ thống sử dụng các kiến trúc học sâu để thực hiện nhận diện và xác thực phương tiện giao thông:

\textbf{Nhận dạng biển số xe:} YOLO được sử dụng để phát hiện vùng biển số, sau đó tiếp tục phân đoạn từng ký tự. Các ký tự được trích xuất được đưa vào mạng CNN để nhận dạng và chuyển đổi thành chuỗi số hóa.

\textbf{Xác thực xe máy:} Ngoài nhận dạng biển số, hệ thống tích hợp DeepFace để nhận diện khuôn mặt người lái, so sánh khuôn mặt lúc vào và ra nhằm ngăn chặn gian lận.

\textbf{Xác thực ô tô:} Sử dụng YOLO để phát hiện và cắt ảnh xe, sau đó áp dụng Siamese Network để so sánh độ tương đồng giữa xe vào và ra. Đồng thời, YOLO cũng được dùng để phát hiện logo thương hiệu làm yếu tố xác thực bổ sung.

Các mô hình được fine-tune trên dữ liệu thu thập tại Việt Nam để thích ứng với đặc thù biển số, góc chụp và điều kiện ánh sáng địa phương.

\subsubsection{Công nghệ IoT}
Hệ thống sử dụng vi điều khiển ESP32 kết nối WiFi để mô phỏng hoạt động của cổng bãi đỗ xe thông minh. ESP32 điều khiển servo SG90 mô phỏng barrier, màn hình LCD hiển thị thông tin, và các button, điện trở để điều khiển thủ công khi cần thiết.

\subsubsection{Xác thực bằng ứng dụng di động}
Thay thế thẻ RFID, hệ thống sử dụng ứng dụng di động liên kết tài khoản người dùng, kết hợp nhận diện biển số tự động để xác thực.

\subsection{Ý nghĩa thực tiễn}

\subsubsection{Đối với người sử dụng}
Người dùng không cần mang thẻ vật lý, sử dụng app di động để vào/ra nhanh chóng. Lịch sử giao dịch được lưu trữ minh bạch, dễ tra cứu. Hệ thống thông báo tự động về trạng thái đỗ xe, nâng cao trải nghiệm.

\subsubsection{Đối với đơn vị quản lý}
Tự động hóa quy trình nhận diện và điều khiển barrier, giảm công việc thủ công. Dashboard cung cấp thống kê thời gian thực về số lượng xe, tình trạng bãi đỗ. Dễ dàng cấp phát và thu hồi quyền truy cập. Giảm chi phí liên quan đến quản lý thẻ RFID.

\subsubsection{Đối với Trường Đại học Giao thông Vận tải TP. HCM}
Ứng dụng công nghệ tiên tiến thể hiện sự hiện đại hóa, giải quyết bài toán quản lý bãi đỗ xe thực tế. Hệ thống nhận diện đa lớp tăng cường an ninh, đồng thời có thể làm mô hình thực hành cho sinh viên về AI và IoT.

\subsubsection{Khả năng mở rộng}
Giải pháp có thể áp dụng cho các trường đại học, khu chung cư, tòa nhà văn phòng, bãi đỗ xe công cộng, góp phần vào xu hướng chuyển đổi số và xây dựng đô thị thông minh.