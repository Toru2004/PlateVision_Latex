\section{Phân tích yêu cầu hệ thống}
    \subsection{Phân tích hiện trạng}

    Hiện nay, phần lớn các bãi giữ xe tại các trường đại học và khu vực công cộng vẫn sử dụng phương pháp quản lý truyền thống dựa trên thẻ từ hoặc thẻ giấy. Quy trình này đòi hỏi sự can thiệp trực tiếp của con người trong việc kiểm soát phương tiện ra vào, dễ phát sinh các vấn đề như thất lạc thẻ, gian lận hoặc ùn tắc vào giờ cao điểm.

    Bên cạnh đó, việc quản lý dữ liệu ra vào chủ yếu được thực hiện thủ công hoặc bán tự động, gây khó khăn trong việc thống kê, truy xuất lịch sử và đảm bảo tính minh bạch. Các hệ thống hiện tại cũng chưa khai thác hiệu quả các công nghệ mới như trí tuệ nhân tạo và IoT để nâng cao hiệu quả vận hành.

    \subsection{Xác định bài toán}

    Từ phân tích hiện trạng, bài toán đặt ra cho đề tài là xây dựng một hệ thống bãi đỗ xe thông minh có khả năng tự động nhận diện và xác thực phương tiện, giảm sự phụ thuộc vào thẻ vật lý và nhân lực vận hành. Hệ thống cần đảm bảo khả năng hoạt động ổn định, chính xác và phù hợp với điều kiện triển khai thực tế tại các bãi giữ xe quy mô vừa và nhỏ.

    Ngoài ra, bài toán còn bao gồm việc tích hợp các thành phần phần cứng IoT, mô hình AI và nền tảng phần mềm quản lý, tạo thành một hệ thống thống nhất, dễ mở rộng và có khả năng phát triển trong tương lai.

    \subsection{Yêu cầu chức năng}

    Hệ thống bãi đỗ xe thông minh cần đáp ứng các yêu cầu chức năng cơ bản sau:

    Thứ nhất, hệ thống phải hỗ trợ tự động nhận diện biển số phương tiện khi xe ra vào bãi đỗ, ghi nhận và đối chiếu thông tin với cơ sở dữ liệu.

    Thứ hai, hệ thống cần điều khiển cổng ra vào tự động thông qua thiết bị IoT, đảm bảo quy trình vận hành nhanh chóng và chính xác.

    Thứ ba, hệ thống phải cung cấp nền tảng quản lý tập trung cho phép quản trị viên theo dõi trạng thái bãi xe, quản lý người dùng và tra cứu lịch sử ra vào.

    Thứ tư, hệ thống cần có ứng dụng di động hỗ trợ người dùng và nhân viên bảo vệ trong việc theo dõi thông tin, nhận thông báo và thực hiện các thao tác cần thiết.

    \subsection{Yêu cầu phi chức năng}

    Bên cạnh các yêu cầu chức năng, hệ thống cần đáp ứng các yêu cầu phi chức năng nhằm đảm bảo chất lượng và khả năng vận hành lâu dài. Hệ thống phải có độ ổn định cao, đảm bảo hoạt động liên tục trong điều kiện mạng không ổn định. Thời gian phản hồi cần đủ nhanh để tránh ùn tắc tại cổng ra vào.

    Ngoài ra, hệ thống cần đảm bảo tính bảo mật dữ liệu người dùng, khả năng mở rộng khi quy mô bãi xe tăng lên, cũng như giao diện thân thiện, dễ sử dụng đối với các đối tượng người dùng khác nhau.
