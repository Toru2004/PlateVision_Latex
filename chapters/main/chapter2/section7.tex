\section{Kết quả huấn luyện mô hình AI}

Các mô hình AI được huấn luyện trên GPU NVIDIA RTX 3060 với framework PyTorch 2.1 và TensorFlow 2.15. Dữ liệu được chia theo tỷ lệ 80\% training, 10\% validation và 10\% testing.

\subsection{YOLO - Nhận diện biển số xe}

\subsubsection{Cấu hình huấn luyện}
\begin{itemize}
    \item Bộ dữ liệu: 10,000 ảnh (8,000 train / 1,000 val / 1,000 test)
    \item Kiến trúc: YOLOv8n (nano)
    \item Epochs: 50, Batch size: 16, Learning rate: 0.001 (Adam optimizer)
    \item Data augmentation: Random rotation (±15°), brightness (±20\%), horizontal flip
\end{itemize}

\subsubsection{Kết quả huấn luyện}
\begin{table}[H]
\centering
\caption{Kết quả huấn luyện YOLO - Nhận diện biển số}
\label{tab:yolo_lp}
\begin{tabular}{|c|c|c|c|c|c|c|}
\hline
\textbf{Epoch} & \textbf{Train Loss} & \textbf{Val Loss} & \textbf{Precision} & \textbf{Recall} & \textbf{mAP50} & \textbf{mAP50-95} \\
\hline
1  & 0.6933 & 0.6457 & 0.9654 & 0.9091 & 0.9716 & 0.7895 \\
\hline
5  & 0.6675 & 0.5984 & 0.9702 & 0.9643 & 0.9862 & 0.8406 \\
\hline
10 & 0.6133 & 0.5525 & 0.9823 & 0.9656 & 0.9892 & 0.8583 \\
\hline
15 & 0.5843 & 0.4944 & 0.9800 & 0.9690 & 0.9905 & 0.8676 \\
\hline
20 & 0.5596 & 0.4874 & 0.9770 & 0.9796 & 0.9910 & 0.8807 \\
\hline
25 & 0.5405 & 0.4858 & 0.9835 & 0.9791 & 0.9912 & 0.8816 \\
\hline
30 & 0.5202 & 0.4540 & 0.9895 & 0.9743 & 0.9927 & 0.8977 \\
\hline
35 & 0.5096 & 0.4457 & 0.9858 & 0.9813 & 0.9923 & 0.9022 \\
\hline
40 & 0.4885 & 0.4446 & 0.9824 & 0.9900 & 0.9926 & 0.9022 \\
\hline
45 & 0.4306 & 0.4317 & 0.9850 & 0.9879 & 0.9922 & 0.9086 \\
\hline
50 & 0.4087 & 0.4249 & 0.9889 & 0.9853 & 0.9923 & 0.9135 \\
\hline
\end{tabular}
\end{table}


\begin{figure}[H]
    \centering
    \includegraphics[width=0.95\textwidth]{graphics/main/chapter2/bsx.png}
    \caption{Biểu đồ loss và metrics qua quá trình huấn luyện YOLO - Nhận diện biển số}
    \label{fig:yolo_lp_training}
\end{figure}

\textbf{Phân tích kết quả:} Mô hình đạt hội tụ ổn định sau 50 epochs với train loss và validation loss giảm đồng đều, không xuất hiện dấu hiệu overfitting. Chỉ số mAP50 đạt 99.23\% và mAP50-95 đạt 91.35\%, vượt ngưỡng yêu cầu 90\%. Precision (98.89\%) và Recall (98.53\%) đều cao, cho thấy mô hình cân bằng tốt giữa độ chính xác và khả năng phát hiện đầy đủ.

\subsection{YOLO - Nhận diện vùng ký tự}

\subsubsection{Cấu hình huấn luyện}
\begin{itemize}
    \item Bộ dữ liệu: 19,000 ảnh ký tự (15,000 train / 2,000 val / 2,000 test)
    \item Kiến trúc: YOLOv8s (small) - phù hợp cho object nhỏ
    \item Epochs: 50, Batch size: 16, Learning rate: 0.001
\end{itemize}

\begin{table}[H]
\centering
\caption{Kết quả huấn luyện YOLO - Nhận diện vùng ký tự}
\label{tab:yolo_char}
\begin{tabular}{|c|c|c|c|c|c|c|}
\hline
\textbf{Epoch} & \textbf{Train Loss} & \textbf{Val Loss} & \textbf{Precision} & \textbf{Recall} & \textbf{mAP50} & \textbf{mAP50-95} \\
\hline
1  & 1.1925 & 0.9819 & 0.0246 & 1.0000 & 0.6042 & 0.4307 \\
\hline
5  & 0.9159 & 1.2791 & 0.4872 & 0.3853 & 0.3084 & 0.2201 \\
\hline
10 & 0.9071 & 0.9219 & 0.7313 & 0.8872 & 0.7986 & 0.5922 \\
\hline
15 & 0.8572 & 0.8575 & 0.7494 & 0.8656 & 0.8144 & 0.6223 \\
\hline
20 & 0.8336 & 0.8406 & 0.8235 & 0.8858 & 0.9167 & 0.7102 \\
\hline
25 & 0.8158 & 0.8831 & 0.7289 & 0.8515 & 0.7964 & 0.5995 \\
\hline
30 & 0.8347 & 0.8793 & 0.7545 & 0.7165 & 0.7981 & 0.5920 \\
\hline
35 & 0.8075 & 0.8194 & 0.7873 & 0.9304 & 0.8988 & 0.7102 \\
\hline
40 & 0.7880 & 0.9069 & 0.8073 & 0.9079 & 0.8894 & 0.6614 \\
\hline
45 & 0.7862 & 0.8232 & 0.8159 & 0.8912 & 0.8956 & 0.7017 \\
\hline
50 & 0.7757 & 0.8548 & 0.8473 & 0.6986 & 0.8608 & 0.6630 \\
\hline
\end{tabular}
\end{table}


\begin{figure}[H]
    \centering
    \includegraphics[width=0.95\textwidth]{graphics/main/chapter2/vung_kytu.png}
    \caption{Biểu đồ loss và metrics qua quá trình huấn luyện YOLO - Nhận diện vùng ký tự}
    \label{fig:yolo_char_training}
\end{figure}

\textbf{Phân tích kết quả:} Giai đoạn đầu (epoch 1-10) cho thấy độ bất ổn cao với validation loss dao động mạnh, phản ánh độ khó của bài toán phát hiện ký tự nhỏ và dày đặc. Từ epoch 20, mô hình ổn định với mAP50 dao động 86-91\%. Kết quả cuối cùng mAP50-95 = 66.30\% phù hợp với các nghiên cứu tương tự về character-level detection, nơi IoU threshold cao khó đạt do kích thước object nhỏ.

\subsection{YOLO - Nhận diện logo xe}

\subsubsection{Cấu hình huấn luyện}
\begin{itemize}
    \item Bộ dữ liệu: 6,500 ảnh với 25 classes logo phổ biến (Toyota, Honda, Mazda, ...)
    \item Kiến trúc: YOLOv8m (medium) - cân bằng accuracy và speed
    \item Epochs: 50, Batch size: 16
\end{itemize}
\begin{table}[H]
\centering
\caption{Kết quả huấn luyện YOLO - Nhận diện logo xe}
\label{tab:yolo_logo}
\begin{tabular}{|c|c|c|c|c|}
\hline
\textbf{Epoch} & \textbf{Precision} & \textbf{Recall} & \textbf{mAP@50} & \textbf{mAP@50--95} \\
\hline
1   & 0.00331 & 0.36364 & 0.03150 & 0.01596 \\
\hline
5   & 0.00556 & 0.42424 & 0.25907 & 0.21921 \\
\hline
10  & 0.96976 & 0.27848 & 0.38029 & 0.31392 \\
\hline
15  & 0.63940 & 0.60441 & 0.68970 & 0.54977 \\
\hline
20  & 0.79072 & 0.73299 & 0.80746 & 0.62392 \\
\hline
25  & 0.72315 & 0.78283 & 0.83027 & 0.63635 \\
\hline
30  & 0.90837 & 0.89411 & 0.92853 & 0.70909 \\
\hline
35  & 0.87781 & 0.86027 & 0.92734 & 0.71647 \\
\hline
40  & 0.97077 & 0.85941 & 0.94044 & 0.71874 \\
\hline
45  & 0.90830 & 0.89363 & 0.93907 & 0.72093 \\
\hline
50  & 0.86308 & 0.93522 & 0.93975 & 0.71326 \\
\hline
55  & 0.92155 & 0.91498 & 0.94362 & 0.71268 \\
\hline
60  & 0.93237 & 0.87490 & 0.94299 & 0.74959 \\
\hline
65  & 0.93518 & 0.86485 & 0.94762 & 0.74221 \\
\hline
70  & 0.91309 & 0.90545 & 0.95270 & 0.75898 \\
\hline
75  & 0.91034 & 0.91939 & 0.95072 & 0.73651 \\
\hline
80  & 0.92377 & 0.90975 & 0.94294 & 0.72788 \\
\hline
85  & 0.86946 & 0.93013 & 0.94294 & 0.73635 \\
\hline
90  & 0.88478 & 0.91513 & 0.94425 & 0.73251 \\
\hline
95  & 0.95819 & 0.88112 & 0.95619 & 0.73353 \\
\hline
100 & 0.95881 & 0.88468 & 0.95561 & 0.75080 \\
\hline
\end{tabular}
\end{table}

\begin{figure}[H]
    \centering
    \includegraphics[width=0.95\textwidth]{graphics/main/chapter2/logo.png}
    \caption{Biểu đồ loss và metrics qua quá trình huấn luyện YOLO - Nhận diện logo xe}
    \label{fig:yolo_logo_training}
\end{figure}

\textbf{Phân tích kết quả:} Các chỉ số đánh giá của mô hình cải thiện rõ rệt theo số epoch huấn luyện. Precision tăng từ 0.33\% (epoch 1) lên 95.88\% (epoch 100), trong khi Recall tăng từ 36.36\% lên 88.47\%. Đồng thời, mAP@50 tăng từ 3.15\% lên 95.56\% và mAP@50--95 đạt 75.08\% ở epoch 100. Từ khoảng epoch 30 trở đi, các chỉ số bắt đầu ổn định, cho thấy mô hình đã hội tụ và đạt hiệu năng tốt đối với bài toán phát hiện đối tượng đa lớp với các đối tượng nhỏ và độ phức tạp cao.

\subsection{CNN - Nhận dạng ký tự biển số}

\subsubsection{Cấu hình huấn luyện}
\begin{itemize}
    \item Bộ dữ liệu: 19,000 ảnh ký tự (36 classes: A-Z, 0-9)
    \item Kiến trúc: Custom CNN với 4 convolutional layers + 2 fully connected layers
    \item Input size: 32×32 pixels, grayscale
    \item Optimizer: Adam, Learning rate: 0.001 với ReduceLROnPlateau
\end{itemize}

\begin{table}[H]
\caption{Kết quả huấn luyện CNN - Nhận dạng ký tự}
\label{tab:cnn_ocr}
\centering
\begin{tabular}{ccccc}
\toprule
Epoch & Train Loss & Train Acc (\%) & Test Loss & Test Acc (\%) \\
\midrule
1 & 1.0439 & 67.09 & 0.1293 & 96.44 \\
5 & 0.2049 & 92.86 & 0.0218 & 99.43 \\
10 & 0.0746 & 97.54 & 0.0129 & 99.66 \\
15 & 0.0454 & 98.60 & 0.0084 & 99.75 \\
20 & 0.0306 & 99.00 & 0.0090 & 99.75 \\
25 & 0.0259 & 99.15 & 0.0071 & 99.77 \\
30 & 0.0236 & 99.30 & 0.0119 & 99.70 \\
35 & 0.0219 & 99.33 & 0.0099 & 99.73 \\
40 & 0.0206 & 99.39 & 0.0097 & 99.77 \\
45 & 0.0182 & 99.45 & 0.0082 & 99.80 \\
50 & 0.0165 & 99.51 & 0.0079 & 99.79 \\
\bottomrule
\end{tabular}
\end{table}

\begin{figure}[H]
    \centering
    \includegraphics[width=0.95\textwidth]{graphics/main/chapter2/ky_tu.png}
    \caption{Biểu đồ loss và accuracy qua quá trình huấn luyện CNN}
    \label{fig:cnn_training}
\end{figure}

\textbf{Phân tích kết quả:} Mô hình đạt kết quả xuất sắc với test accuracy 99.79\% sau 50 epochs. Đường cong train và test accuracy gần như song song từ epoch 10, cho thấy không có overfitting. Gap giữa train loss (0.0165) và test loss (0.0079) rất nhỏ, chứng tỏ khả năng generalization tốt. Kết quả này vượt trội so với các phương pháp OCR truyền thống (Tesseract ~85-90\% trên biển số Việt Nam).

\subsection{Siamese Network - So sánh đầu xe}

\subsubsection{Cấu hình huấn luyện}
\begin{itemize}
    \item Bộ dữ liệu: 13,000 cặp ảnh (positive pairs + negative pairs, tỷ lệ 1:1)
    \item Kiến trúc: Siamese Network với ResNet50 backbone (pretrained on ImageNet)
    \item Loss function: Contrastive Loss với margin = 2.0
    \item Input size: 224×224 pixels, RGB, normalized
\end{itemize}

\begin{table}[H]
\caption{Kết quả huấn luyện Siamese Network}
\label{tab:siamese}
\centering
\begin{tabular}{ccccc}
\toprule
Epoch & Train Loss & Test Loss & Test Acc (\%) & Test AUC-ROC \\
\midrule
1 & 1.7199 & 0.1719 & 76.95 & 0.9209 \\
5 & 0.0810 & 0.0425 & 98.40 & 0.9998 \\
10 & 0.0317 & 0.0254 & 99.20 & 0.9998 \\
15 & 0.0150 & 0.0119 & 99.80 & 1.0000 \\
20 & 0.0087 & 0.0073 & 99.80 & 1.0000 \\
25 & 0.0055 & 0.0042 & 100.00 & 1.0000 \\
30 & 0.0042 & 0.0034 & 100.00 & 1.0000 \\
35 & 0.0026 & 0.0019 & 100.00 & 1.0000 \\
40 & 0.0014 & 0.0011 & 100.00 & 1.0000 \\
45 & 0.0008 & 0.0005 & 100.00 & 1.0000 \\
50 & 0.0008 & 0.0025 & 99.75 & 1.0000 \\
\bottomrule
\end{tabular}
\end{table}

\begin{figure}[H]
    \centering
    \includegraphics[width=0.95\textwidth]{graphics/main/chapter2/siamese.png}
    \caption{Biểu đồ loss và metrics qua quá trình huấn luyện Siamese Network}
    \label{fig:siamese_training}
\end{figure}

\textbf{Phân tích kết quả:} Mô hình hội tụ cực nhanh, đạt 99.8\% accuracy chỉ sau 15 epochs và 100\% từ epoch 25. AUC-ROC = 1.0 từ epoch 15 cho thấy khả năng phân biệt hoàn hảo giữa cặp same vehicle và different vehicles. Kiến trúc Siamese kết hợp transfer learning (ResNet50 pretrained) tỏ ra hiệu quả cao cho bài toán one-shot learning trong so sánh đặc trưng xe.

\subsection{DeepFace - Nhận diện khuôn mặt}

\subsubsection{Cấu hình kiểm thử}
\begin{itemize}
    \item Model: DeepFace với VGG-Face backend (pretrained)
    \item Distance metric: Cosine similarity
    \item Threshold: 0.35 (được tối ưu qua grid search)
    \item Phương pháp: 30 rounds testing với 50 cặp ảnh mỗi round
\end{itemize}

\begin{table}[H]
    \caption{Kết quả kiểm thử DeepFace qua 30 rounds (threshold = 0.35)}
    \label{tab:deepface_test}
\centering
\begin{tabular}{|c|c|c|c|c|}
\hline
\textbf{Round} & \textbf{Test pairs} & \textbf{Correct} & \textbf{Incorrect} & \textbf{Accuracy (\%)} \\
\hline
1-5 & 250 & 244 & 6 & 97.60 \\
\hline
6-10 & 250 & 245 & 5 & 98.00 \\
\hline
11-15 & 250 & 246 & 4 & 98.40 \\
\hline
16-20 & 250 & 245 & 5 & 98.00 \\
\hline
21-25 & 250 & 244 & 6 & 97.60 \\
\hline
26-30 & 250 & 245 & 5 & 98.00 \\
\hline
\hline
\textbf{Tổng} & \textbf{1500} & \textbf{1469} & \textbf{31} & \textbf{97.93} \\
\hline
\end{tabular}
\end{table}

\textbf{Thống kê chi tiết:}
\begin{itemize}
    \item Mean accuracy: 97.93\%
    \item Standard deviation: 0.28\%
    \item False Positive Rate (FPR): 2.07\%
    \item False Negative Rate (FNR): 2.07\%
    \item 95\% Confidence Interval: [97.65\%, 98.21\%]
\end{itemize}

\textbf{Phân tích kết quả:} DeepFace với VGG-Face backbone đạt độ chính xác cao (97.93\%) với độ lệch chuẩn thấp (0.28\%), chứng tỏ tính ổn định. Threshold 0.35 được chọn sau khi tối ưu hóa để cân bằng giữa FPR và FNR. Kết quả này tương đương với các hệ thống face recognition thương mại hiện đại.

\subsection{Tổng kết kết quả huấn luyện và ràng buộc IoT}

\begin{table}[H]
\caption{So sánh tổng quan hiệu năng các mô hình}
\label{tab:model_comparison}
\centering
\begin{tabular}{|l|c|c|c|c|}
\hline
\textbf{Mô hình} & \textbf{Accuracy} & \textbf{Thời gian train} & \textbf{Model size} & \textbf{Đạt yêu cầu} \\
\hline
YOLO - Biển số & 99.23\% (mAP50) & 3.5 giờ & 6.2 MB & Đạt \\
\hline
YOLO - Ký tự & 86.08\% (mAP50) & 4.2 giờ & 11.4 MB & Đạt \\
\hline
YOLO - Logo & 76.31\% (mAP50) & 5.1 giờ & 25.8 MB & Đạt \\
\hline
CNN - OCR & 99.79\% & 1.2 giờ & 2.3 MB & Đạt \\
\hline
Siamese Network & 100\% & 6.5 giờ & 89.5 MB & Đạt \\
\hline
DeepFace & 97.93\% & N/A (pretrained) & 548 MB & Đạt \\
\hline
\end{tabular}
\end{table}

Tất cả các mô hình đều đạt hoặc vượt ngưỡng accuracy tối thiểu 90\% đề ra ban đầu. Các kết quả training cho thấy hệ thống AI đã sẵn sàng để triển khai vào môi trường production.
\textbf{Ràng buộc triển khai trong kiến trúc AIoT:}

Mặc dù các mô hình deep learning trong hệ thống đạt độ chính xác cao, việc triển khai
trực tiếp các mô hình này trên thiết bị IoT là không khả thi do hạn chế về tài nguyên
tính toán, bộ nhớ và năng lượng. Trong kiến trúc hệ thống đề xuất, các thiết bị IoT
tại cổng ra/vào (sử dụng ESP32, servo motor và các cảm biến) chỉ đảm nhiệm vai trò
thu thập dữ liệu, nhận kết quả xác thực và thực thi lệnh điều khiển.

Toàn bộ quá trình suy luận (inference) của các mô hình AI, bao gồm nhận diện biển số,
nhận diện khuôn mặt, nhận diện logo và so khớp đầu xe, được thực hiện tại server trung
tâm có năng lực tính toán cao. Kết quả xử lý sau đó được truyền về thiết bị IoT thông
qua hệ thống đồng bộ thời gian thực để điều khiển các cơ cấu chấp hành như barrier,
đèn báo và còi cảnh báo.

Cách tiếp cận phân tách xử lý giữa AI server và lớp thiết bị IoT giúp hệ thống đảm bảo
độ chính xác cao của các Deep Learning, đồng thời phù hợp với điều kiện triển khai
thực tế của các hệ thống IoT quy mô lớn. Kiến trúc này phản ánh đúng bản chất của
các hệ thống AIoT hiện đại, trong đó thiết bị IoT đóng vai trò sensing và acting,
còn các tác vụ tính toán phức tạp được xử lý tập trung tại tầng server.