\section{Thiết kế chi tiết}

\subsection{Module camera quét cổng ra/vào}

Module camera quét cổng ra/vào được xây dựng nhằm tự động nhận diện phương tiện và xác thực thông tin khi xe vào/ra bãi đỗ. Module này đóng vai trò nền tảng trong việc định danh chính xác phương tiện, đảm bảo an ninh và phục vụ cho các chức năng kiểm soát truy cập cũng như quản lý lịch sử đỗ xe.

\subsubsection{Quy trình xử lý chung}

Module hoạt động theo luồng xử lý phân nhánh dựa trên loại phương tiện (xe máy hoặc ô tô), bao gồm các bước chính:

\textbf{Giai đoạn 1: Nhận diện biển số (chung cho cả xe máy và ô tô)}
\begin{itemize}
    \item Thu nhận hình ảnh từ camera giám sát tại cổng vào/ra;
    \item Phát hiện vùng chứa biển số bằng mô hình YOLOv8 (License Plate Detection);
    \item Phát hiện vùng ký tự trên biển số bằng mô hình YOLOv8 (Character Region Detection);
    \item Nhận dạng từng ký tự bằng mô hình CNN (Character Recognition);
    \item Tổng hợp chuỗi ký tự thành biển số hoàn chỉnh kèm theo độ tin cậy.
\end{itemize}

\textbf{Giai đoạn 2: Xác thực phương tiện (phân nhánh theo loại xe)}

\textbf{2.1. Xử lý cho xe máy:}
\begin{itemize}
    \item \textbf{Khi xe vào:}
    \begin{itemize}
        \item Chụp ảnh khuôn mặt người lái xe;
        \item Trích xuất embedding khuôn mặt bằng DeepFace (VGG-Face backend);
        \item Lưu trữ embedding cùng biển số và thời gian vào hệ thống cơ sở dữ liệu.
    \end{itemize}
    
    \item \textbf{Khi xe ra:}
    \begin{itemize}
        \item Chụp ảnh khuôn mặt người lái hiện tại;
        \item Trích xuất embedding khuôn mặt mới;
        \item Truy xuất embedding đã lưu dựa trên biển số;
        \item So sánh hai embedding bằng Cosine Similarity với ngưỡng $threshold = 0.35$;
        \item Nếu similarity $>$ threshold $\rightarrow$ cho phép xe ra;
        \item Ngược lại $\rightarrow$ kích hoạt cơ chế cảnh báo.
    \end{itemize}
\end{itemize}

\textbf{2.2. Xử lý cho ô tô:}
\begin{itemize}
    \item \textbf{Khi xe vào:}
    \begin{itemize}
        \item Nhận diện logo xe bằng mô hình YOLOv8;
        \item Chụp ảnh toàn cảnh đầu xe;
        \item Trích xuất vector đặc trưng bằng Siamese Network (ResNet50 backbone);
        \item Lưu trữ logo, vector đặc trưng, ảnh đầu xe, biển số và thời gian.
    \end{itemize}
    
    \item \textbf{Khi xe ra:}
    \begin{itemize}
        \item Nhận diện logo xe hiện tại;
        \item Trích xuất vector đặc trưng mới;
        \item Truy xuất dữ liệu đã lưu theo biển số;
        \item So sánh logo và tính khoảng cách Euclidean giữa hai vector;
        \item Nếu logo khớp và khoảng cách nhỏ hơn ngưỡng $\rightarrow$ cho phép xe ra;
        \item Ngược lại $\rightarrow$ gửi cảnh báo an ninh.
    \end{itemize}
\end{itemize}

\subsection{Module quản lý bãi đỗ}

Module quản lý bãi đỗ chịu trách nhiệm điều phối toàn bộ hoạt động của hệ thống, bao gồm quản lý vị trí đỗ, trạng thái chỗ trống, lịch sử ra/vào và thông tin phương tiện. Module này đóng vai trò trung tâm trong việc tổng hợp dữ liệu từ các module nhận diện và IoT.

Dữ liệu được lưu trữ trong cơ sở dữ liệu tập trung, hỗ trợ truy vấn, thống kê và xuất báo cáo. Ngoài ra, module cung cấp các API để web quản trị và ứng dụng mobile theo dõi tình trạng bãi đỗ theo thời gian thực.

\subsection{Ứng dụng mobile}

Ứng dụng mobile được phát triển nhằm cung cấp giao diện tương tác trực tiếp cho người dùng cuối. Thông qua ứng dụng, người dùng có thể đăng ký tài khoản, quản lý phương tiện, theo dõi trạng thái xe và nhận thông báo thời gian thực về các sự kiện ra/vào hoặc bất thường.

Ứng dụng giao tiếp với backend thông qua các API bảo mật, đảm bảo an toàn và toàn vẹn dữ liệu.

\subsection{Module mô phỏng phần cứng IoT}

Module mô phỏng phần cứng IoT được xây dựng nhằm tái hiện hoạt động của các thiết bị IoT trong bãi đỗ xe thông minh, trong điều kiện nghiên cứu và thử nghiệm với chi phí thấp.

\subsubsection{Vai trò của lớp thiết bị IoT trong hệ thống}

Lớp thiết bị IoT đảm nhiệm các chức năng:
\begin{itemize}
    \item Thu thập dữ liệu từ các cảm biến;
    \item Nhận lệnh điều khiển từ hệ thống trung tâm;
    \item Thực thi hành động vật lý và phản hồi trạng thái hoạt động.
\end{itemize}

\subsubsection{Lý do lựa chọn ESP32 để mô phỏng}

ESP32 được lựa chọn do có đầy đủ các đặc tính tương đồng với thiết bị IoT công nghiệp như khả năng kết nối mạng, xử lý thời gian thực và điều khiển ngoại vi. Việc sử dụng ESP32 giúp mô phỏng chính xác hành vi của hệ thống thực tế trong khi vẫn phù hợp với điều kiện nghiên cứu.

\subsubsection{Mô phỏng thiết bị công nghiệp}

\begin{center}
\begin{tabular}{|c|c|}
\hline
\textbf{Linh kiện} & \textbf{Thiết bị công nghiệp đại diện} \\
\hline
Servo SG90 & Barie tự động \\
LCD & Bảng hiển thị thông tin \\
Cảm biến rung & Cảm biến va chạm \\
Còi & Hệ thống cảnh báo \\
Button & Nút khẩn cấp \\
\hline
\end{tabular}
\end{center}

\subsubsection{Phân tách xử lý AI và IoT}

Do hạn chế về tài nguyên tính toán của vi điều khiển, các mô hình deep learning không được triển khai trực tiếp trên ESP32 mà được xử lý tại server trung tâm. Thiết bị IoT chỉ đảm nhận vai trò thu thập dữ liệu và thực thi quyết định điều khiển, phù hợp với kiến trúc AIoT trong thực tế.

\subsection{Tích hợp các module}

Các module trong hệ thống được tích hợp theo kiến trúc phân tán, trong đó Firebase đóng vai trò trung tâm đồng bộ dữ liệu. Mỗi module đảm nhiệm một chức năng riêng biệt nhưng có khả năng phối hợp chặt chẽ thông qua các API và cơ chế đồng bộ thời gian thực.

Thiết kế này giúp hệ thống linh hoạt, dễ mở rộng và cho phép thay thế hoặc nâng cấp từng module mà không ảnh hưởng đến toàn bộ kiến trúc.
