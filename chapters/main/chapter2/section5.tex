\section{Thiết kế chi tiết}

\subsection{Module camera quét cổng ra/vào}

Module camera quét cổng ra/vào được xây dựng nhằm tự động nhận diện phương tiện và xác thực thông tin khi xe vào/ra bãi đỗ. Module này đóng vai trò nền tảng trong việc định danh chính xác phương tiện, đảm bảo an ninh và phục vụ cho các chức năng kiểm soát truy cập cũng như quản lý lịch sử đỗ xe.

\subsubsection{Quy trình xử lý chung}

Module hoạt động theo luồng xử lý phân nhánh dựa trên loại phương tiện (xe máy hoặc ô tô), bao gồm các bước chính:

\textbf{Giai đoạn 1: Nhận diện biển số (chung cho cả xe máy và ô tô)}
\begin{itemize}
    \item Thu nhận hình ảnh từ camera giám sát tại cổng vào/ra
    \item Phát hiện vùng chứa biển số bằng mô hình YOLOv8 (License Plate Detection)
    \item Phát hiện vùng ký tự trên biển số bằng mô hình YOLOv8 (Character Region Detection)
    \item Nhận dạng từng ký tự bằng mô hình CNN (Character Recognition)
    \item Tổng hợp chuỗi ký tự thành biển số hoàn chỉnh với độ tin cậy
\end{itemize}

\textbf{Giai đoạn 2: Xác thực phương tiện (phân nhánh theo loại xe)}

\textbf{2.1. Xử lý cho xe máy:}
\begin{itemize}
    \item \textbf{Khi xe vào:}
    \begin{itemize}
        \item Chụp ảnh khuôn mặt người lái xe từ camera
        \item Trích xuất embedding khuôn mặt bằng DeepFace (VGG-Face backend)
        \item Lưu trữ embedding cùng với biển số và timestamp vào database
    \end{itemize}
    
    \item \textbf{Khi xe ra:}
    \begin{itemize}
        \item Chụp ảnh khuôn mặt người lái hiện tại
        \item Trích xuất embedding khuôn mặt mới
        \item Truy xuất embedding đã lưu từ database dựa trên biển số
        \item So sánh hai embedding bằng Cosine Similarity với threshold = 0.35
        \item Xác nhận: Nếu similarity > threshold → Cùng người lái → Cho phép ra
        \item Cảnh báo: Nếu similarity $\le$ threshold $\rightarrow$ Khác người lái $\rightarrow$ Gửi cảnh báo

    \end{itemize}
\end{itemize}

\textbf{2.2. Xử lý cho ô tô:}
\begin{itemize}
    \item \textbf{Khi xe vào:}
    \begin{itemize}
        \item Nhận diện logo xe bằng mô hình YOLOv8 (Logo Detection)
        \item Chụp ảnh toàn cảnh đầu xe (góc chính diện)
        \item Trích xuất feature vector từ ảnh đầu xe bằng Siamese Network (ResNet50 backbone)
        \item Lưu trữ: logo, feature vector, ảnh đầu xe, biển số, timestamp vào database
    \end{itemize}
    
    \item \textbf{Khi xe ra:}
    \begin{itemize}
        \item Nhận diện logo xe hiện tại
        \item Chụp ảnh đầu xe hiện tại
        \item Trích xuất feature vector mới từ ảnh đầu xe
        \item Truy xuất thông tin đã lưu từ database dựa trên biển số
        \item \textbf{Bước 1 - So sánh logo:} Kiểm tra logo hiện tại có khớp với logo đã lưu
        \item \textbf{Bước 2 - So sánh đầu xe:} Tính khoảng cách Euclidean giữa hai feature vectors
        \item Xác nhận: Nếu logo khớp VÀ distance < threshold → Cùng xe → Cho phép ra
        \item Cảnh báo: Nếu logo khác HOẶC distance $\ge$ threshold $\rightarrow$ Khác xe $\rightarrow$ Gửi cảnh báo

    \end{itemize}
\end{itemize}

\subsection{Module quản lý bãi đỗ}
Module quản lý bãi đỗ chịu trách nhiệm điều phối toàn bộ hoạt động trong bãi xe, bao gồm quản lý vị trí đỗ, trạng thái chỗ trống, lịch sử ra/vào và thông tin phương tiện. Module này đóng vai trò trung tâm trong việc tổng hợp và xử lý dữ liệu từ các module nhận diện.

Dữ liệu được lưu trữ trong cơ sở dữ liệu tập trung, hỗ trợ các thao tác truy vấn, thống kê và báo cáo. Module cho phép cấu hình số lượng khu vực, số chỗ đỗ, cũng như các quy tắc vận hành (thời gian đỗ tối đa, cảnh báo quá hạn).

Ngoài ra, module còn cung cấp các API để web admin và ứng dụng mobile có thể truy cập, theo dõi tình trạng bãi xe theo thời gian thực.

\subsection{Ứng dụng mobile}
Ứng dụng mobile được phát triển nhằm cung cấp giao diện tương tác trực tiếp cho người dùng cuối. Thông qua ứng dụng, người dùng có thể đăng ký tài khoản, quản lý thông tin cá nhân, đăng ký phương tiện và theo dõi trạng thái xe trong bãi đỗ.

Ứng dụng hỗ trợ nhận thông báo thời gian thực (push notification) về các sự kiện quan trọng như xe vào/ra bãi, sắp hết thời gian đỗ hoặc phát hiện bất thường. Giao diện được thiết kế theo hướng thân thiện, dễ sử dụng và phù hợp với nhiều đối tượng người dùng.

Về mặt kỹ thuật, ứng dụng mobile giao tiếp với hệ thống backend thông qua các API bảo mật, đảm bảo tính toàn vẹn và an toàn của dữ liệu.
\subsection{Module mô phỏng phần cứng IoT}
Module mô phỏng phần cứng IoT được xây dựng để mô phỏng hoạt động của các thiết bị cảm biến và vi điều khiển trong môi trường thực tế. Module này chịu trách nhiệm nhận dữ liệu từ firebase để điều khiển các thiết bị mô phỏng như đèn báo, cổng tự động.

\subsection{Tích hợp các module}
Các module trong hệ thống được tích hợp theo kiến trúc phân tán, trong đó mỗi module đảm nhiệm một chức năng riêng biệt nhưng có khả năng phối hợp chặt chẽ với nhau. Việc giao tiếp giữa các module được thực hiện thông qua API hoặc cơ chế message queue nhằm đảm bảo tính mở rộng và khả năng chịu lỗi.

Luồng xử lý tổng thể bắt đầu từ việc camera ghi nhận hình ảnh, sau đó dữ liệu được chuyển đến các module nhận diện để phân tích. Kết quả nhận diện được gửi về module quản lý bãi đỗ để cập nhật trạng thái hệ thống và đồng bộ với ứng dụng mobile cũng như web admin.

Thiết kế tích hợp này giúp hệ thống hoạt động linh hoạt, dễ dàng nâng cấp từng module riêng lẻ mà không ảnh hưởng đến toàn bộ kiến trúc.
