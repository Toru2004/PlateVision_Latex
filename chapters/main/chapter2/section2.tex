\section{Phương pháp luận nghiên cứu}
    \subsection{Phương pháp nghiên cứu thiết kế (Design Science Research)}

    Đề tài được thực hiện theo phương pháp nghiên cứu thiết kế (Design Science Research – DSR), một phương pháp nghiên cứu phổ biến trong lĩnh vực hệ thống thông tin và công nghệ thông tin. Phương pháp này tập trung vào việc xây dựng và đánh giá các giải pháp công nghệ nhằm giải quyết những vấn đề thực tiễn cụ thể.

    Trong bối cảnh đề tài, phương pháp DSR cho phép nhóm tiếp cận bài toán quản lý bãi đỗ xe theo hướng thiết kế một hệ thống tích hợp giữa IoT và trí tuệ nhân tạo, thay vì chỉ dừng lại ở việc phân tích lý thuyết. Sản phẩm nghiên cứu không chỉ là mô hình khái niệm mà còn là một hệ thống thử nghiệm có khả năng vận hành, được kiểm chứng thông qua các tiêu chí đánh giá rõ ràng.

    Việc áp dụng Design Science Research giúp đảm bảo rằng mô hình bãi đỗ xe thông minh được xây dựng có cơ sở khoa học, đồng thời đáp ứng được các yêu cầu thực tiễn trong quá trình triển khai và sử dụng.

    \subsection{Quy trình nghiên cứu}

    Quy trình nghiên cứu của đề tài được thực hiện theo các bước chính sau:

    Thứ nhất, khảo sát và phân tích hiện trạng. Nhóm tiến hành khảo sát thực tế tại bãi giữ xe trong khuôn viên Trường Đại học Giao thông Vận tải TP. Hồ Chí Minh, đồng thời thu thập ý kiến người dùng nhằm đánh giá những hạn chế của mô hình quản lý bãi xe truyền thống.

    Thứ hai, xác định bài toán và mục tiêu nghiên cứu. Từ kết quả khảo sát, nhóm xác định bài toán cần giải quyết là tự động hóa quy trình ra vào bãi đỗ xe, giảm phụ thuộc vào thẻ vật lý và nâng cao mức độ an toàn, minh bạch trong quản lý.

    Thứ ba, đề xuất mô hình hệ thống. Dựa trên các yêu cầu đặt ra, nhóm xây dựng mô hình hệ thống bãi đỗ xe thông minh tích hợp AI và IoT, bao gồm các thành phần phần cứng, phần mềm và các mô-đun xử lý thông minh.

    Thứ tư, thiết kế và triển khai mô hình thử nghiệm. Hệ thống được hiện thực hóa dưới dạng prototype với các chức năng cốt lõi như nhận diện biển số, điều khiển cổng tự động, quản lý dữ liệu và ứng dụng di động.

    Cuối cùng, đánh giá và kiểm định. Mô hình được kiểm thử thông qua các kịch bản thực nghiệm nhằm đánh giá hiệu năng, độ chính xác và tính ổn định của hệ thống so với mục tiêu ban đầu.

    \subsection{Tiêu chí đánh giá}

    Để đánh giá hiệu quả của mô hình nghiên cứu, nhóm sử dụng các tiêu chí đánh giá cả về mặt kỹ thuật và chức năng nghiệp vụ. Về mặt kỹ thuật, các tiêu chí bao gồm độ chính xác của mô hình nhận diện, thời gian phản hồi của hệ thống và khả năng đồng bộ dữ liệu giữa các thành phần. Về mặt chức năng, hệ thống được đánh giá dựa trên mức độ đáp ứng các yêu cầu nghiệp vụ như tự động hóa quy trình ra vào, khả năng quản lý và giám sát, cũng như trải nghiệm người dùng.

    Các tiêu chí này được sử dụng làm cơ sở cho quá trình kiểm định và phân tích kết quả ở các chương tiếp theo, nhằm đảm bảo tính khách quan và khoa học của nghiên cứu.

