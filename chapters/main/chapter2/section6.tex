\section{Kết quả khảo sát thực tế}


Để đánh giá nhu cầu và mức độ cần thiết của hệ thống, nhóm nghiên cứu đã tiến hành khảo sát 151 người dùng tiềm năng về thói quen sử dụng dịch vụ gửi xe và vấn đề mất thẻ xe.


\subsubsection{Tần suất mất thẻ xe}

\begin{figure}[H]
    \centering
    \includegraphics[width=0.7\textwidth]{graphics/main/chapter2/the_xe.jpg}
    \caption{Thời gian gần nhất người dùng bị mất thẻ xe}
    \label{fig:survey_q3}
\end{figure}

Trong số những người từng mất thẻ, 78.1\% không nhớ hoặc đã rất lâu không gặp vấn đề này, 10.6\% mất thẻ trong vòng 1-6 tháng trước, và 6\% mất thẻ trong tháng gần đây. Kết quả này cho thấy mặc dù không phổ biến, nhưng vấn đề mất thẻ vẫn xảy ra và gây phiền toái cho người dùng.

\subsubsection{Mức độ sử dụng dịch vụ gửi xe không cần thẻ}

\begin{figure}[H]
    \centering
    \includegraphics[width=0.7\textwidth]{graphics/main/chapter2/dv.jpg}
    \caption{Mức độ mong muốn sử dụng dịch vụ gửi xe không cần thẻ}
    \label{fig:survey_q1}
\end{figure}

Kết quả cho thấy 44.4\% người dùng rất muốn sử dụng hệ thống gửi xe không cần thẻ (quét QR, nhận diện biển số), trong khi 46.4\% cho rằng có thể sử dụng nếu tiện lợi. Chỉ 9.3\% không quan tâm đến giải pháp này. Điều này chứng tỏ có nhu cầu thực tế cao (90.8\%) đối với hệ thống tự động hóa.


\subsubsection{Kết luận từ khảo sát}

\begin{itemize}
    \item \textbf{Nhu cầu cao:} Hơn 90\% người dùng muốn sử dụng hệ thống gửi xe tự động không cần thẻ
    \item \textbf{Vấn đề mất thẻ tồn tại:} 23.2\% người dùng từng gặp phải, cần giải pháp thay thế
    \item \textbf{Tiềm năng ứng dụng:} Hệ thống đề xuất đáp ứng đúng nhu cầu thực tế của người dùng
\end{itemize}

Kết quả khảo sát này khẳng định tính cần thiết và khả thi của việc xây dựng hệ thống bãi đỗ xe thông minh tích hợp AI, loại bỏ hoàn toàn thẻ xe truyền thống.
