\section{Tiêu chí đánh giá và kiểm định}

Để đảm bảo hệ thống "Bãi đỗ xe thông minh tích hợp AI" vận hành ổn định và đạt được các mục tiêu đề ra, quá trình đánh giá và kiểm định được thực hiện dựa trên các nhóm tiêu chuẩn định lượng về kỹ thuật và định tính về chức năng nghiệp vụ.

\subsection{Tiêu chí đánh giá hiệu năng kỹ thuật}
Hiệu năng kỹ thuật của hệ thống được đánh giá dựa trên độ chính xác của các mô hình học máy và khả năng đáp ứng thời gian thực của hạ tầng IoT:
\begin{itemize}
    \item \textbf{Độ chính xác nhận diện (Accuracy):} Đo lường tỷ lệ nhận diện đúng biển số xe bằng mô hình YOLO, nhận diện khuôn mặt và so khớp logo phương tiện thông qua mạng Siamese trong các điều kiện môi trường khác nhau.
    \item \textbf{Thời gian phản hồi (Latency):} Hệ thống phải đảm bảo khả năng xử lý hình ảnh và điều khiển cổng tự động (barrier) theo thời gian thực để đáp ứng mật độ xe vừa và cao theo thời gian thực để tránh ùn tắc trong giờ cao điểm.
    \item \textbf{Hiệu suất đồng bộ dữ liệu:} Đánh giá tốc độ truyền tải thông tin giữa thiết bị ESP32 và Realtime Database, cũng như khả năng truy xuất hình ảnh từ dịch vụ đám mây Cloudinary.
    \item \textbf{Tính ổn định vận hành:} Khả năng duy trì hoạt động liên tục của phần cứng IoT và tính nhất quán của dữ liệu khi xảy ra các tình trạng mạng yếu hoặc trễ kết nối.
\end{itemize}

\subsection{Tiêu chí đánh giá chức năng}
Hệ thống được kiểm tra dựa trên mức độ hoàn thiện của các quy trình nghiệp vụ đã thiết lập:
\begin{itemize}
    \item \textbf{Tự động hóa quy trình ra/vào:} Đánh giá khả năng tự vận hành của trạm quét mà không cần sự can thiệp trực tiếp từ con người.
    \item \textbf{Cơ chế cảnh báo an ninh:} Kiểm chứng tính hiệu quả của chức năng phát hiện xe rời bãi trái phép và khả năng gửi thông báo tức thời (Push Notification) đến ứng dụng di động của người dùng.
    \item \textbf{Tích hợp thanh toán điện tử:} Xác minh tính chính xác của giao dịch qua cổng VNPAY, đảm bảo tính tiện lợi và rút ngắn thời gian chờ đợi so với thanh toán tiền mặt.
    \item \textbf{Quản lý và thống kê:} Đánh giá giao diện Web Admin trong việc cung cấp số liệu trực quan về doanh thu, lượt xe và xử lý yêu cầu từ bảo vệ.
\end{itemize}

\subsection{Phương pháp thực nghiệm}
Quá trình kiểm định được triển khai thông qua phương pháp thực nghiệm kết hợp giữa mô phỏng và lắp đặt thực tế:
\begin{itemize}
    \item \textbf{Mô phỏng phần cứng:} Sử dụng phần mềm Proteus để kiểm chứng nguyên lý mạch điện tử và luồng tín hiệu của các thiết bị IoT trước khi chế tạo bản mẫu.
    \item \textbf{Thử nghiệm hiện trường:} Lắp ráp các linh kiện thực tế như ESP32, Arduino Uno, động cơ Servo SG90 và cảm biến siêu âm để đánh giá khả năng vận hành trong điều kiện thực tế.
    \item \textbf{Xây dựng kịch bản kiểm thử:} 
    \begin{itemize}
        \item \textit{Kịch bản lý tưởng:} Kiểm thử nhận diện trong điều kiện ánh sáng đầy đủ và biển số rõ nét.
        \item \textit{Kịch bản bất lợi:} Thử nghiệm hệ thống với biển số bị mờ, bùn đất che phủ, hoặc người điều khiển đeo khẩu trang để đánh giá giới hạn sai số của mô hình AI.
    \end{itemize}
\end{itemize}