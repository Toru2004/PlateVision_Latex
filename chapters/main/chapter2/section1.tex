\section{Giới thiệu}

Chương này trình bày quá trình xây dựng mô hình nghiên cứu cho đề tài “Bãi đỗ xe thông minh tích hợp AI”, đóng vai trò làm cầu nối giữa cơ sở lý thuyết và quá trình thiết kế – triển khai hệ thống. Trên cơ sở các vấn đề thực tiễn đã được phân tích ở chương trước, chương 2 tập trung làm rõ phương pháp nghiên cứu được lựa chọn, cách tiếp cận xây dựng mô hình hệ thống cũng như các yêu cầu kỹ thuật và nghiệp vụ cần đáp ứng.

Nội dung chương bao gồm việc trình bày phương pháp luận nghiên cứu theo hướng thiết kế – phát triển hệ thống, phân tích hiện trạng các bãi giữ xe truyền thống, xác định bài toán nghiên cứu và đề xuất các yêu cầu chức năng, phi chức năng cho hệ thống bãi đỗ xe thông minh. Trên cơ sở đó, mô hình hệ thống được xây dựng nhằm đảm bảo tính khả thi, tính khoa học và khả năng ứng dụng trong thực tế.

Kết quả của chương này là nền tảng quan trọng để nhóm triển khai thiết kế kiến trúc hệ thống, xây dựng các module chức năng và tiến hành mô phỏng, thực nghiệm ở các chương tiếp theo.