\section{Phạm vi nghiên cứu}

Đề tài ``Bãi đỗ xe thông minh'' tập trung nghiên cứu và xây dựng mô hình hệ thống quản lý bãi đỗ xe tự động, hỗ trợ giám sát -- đặt chỗ -- nhận diện biển số phương tiện. Phạm vi nghiên cứu của đề tài bao gồm các nội dung chính sau:

\begin{itemize}
    \item Hệ thống phần cứng IoT tại bãi đỗ xe:
    \begin{itemize}
        \item Thiết kế mô hình bãi đỗ xe thu nhỏ hoặc hệ thống thử nghiệm.
        \item Sử dụng các cảm biến siêu âm, RFID/thẻ từ, camera để phát hiện phương tiện và trạng thái chỗ đỗ.
        \item Kết nối thiết bị với vi điều khiển (ESP32/Arduino hoặc tương đương) để truyền dữ liệu lên máy chủ.
    \end{itemize}

    \item Nền tảng quản lý tập trung (Web Admin):
    \begin{itemize}
        \item Theo dõi tình trạng bãi xe theo thời gian thực.
        \item Thêm/xóa/chỉnh sửa thông tin xe, người dùng.
        \item Xem báo cáo thống kê doanh thu.
        \item Hệ thống backend cung cấp API xử lý dữ liệu từ IoT và ứng dụng người dùng.
    \end{itemize}

    \item Ứng dụng di động cho người dùng:
    \begin{itemize}
        \item Xem thông tin tài khoản, nội dung quy định của hệ thống.
        \item Xem trạng thái, check-in/check-out phương tiện.
        \item Xem lịch sử gửi xe, thông báo trạng thái.
    \end{itemize}

    \item Hệ thống AI nhận diện biển số xe (LPR -- License Plate Recognition):
    \begin{itemize}
        \item Triển khai mô hình nhận diện biển số bằng YOLO + CRNN.
        \item Tự động ghi nhận dữ liệu biển số và đối chiếu thông tin với hệ thống quản lý.
    \end{itemize}

    \item Giới hạn phạm vi của đề tài:
    \begin{itemize}
        \item Hệ thống được xây dựng ở mức prototype, quy mô thử nghiệm.
        \item Chưa xử lý nâng cao như biển số mờ, điều kiện ánh sáng kém, quy mô lớn ngoài thực tế.
        \item Tính năng thanh toán trực tuyến mới dừng lại ở mức mô phỏng tích hợp cơ bản.
        \item Hệ thống bảo mật ở mức cơ bản, chưa đạt tiêu chuẩn thương mại hóa.
    \end{itemize}
\end{itemize}

Phạm vi nghiên cứu tập trung vào việc mô hình hóa hệ thống, xây dựng phiên bản thử nghiệm khả thi, đánh giá khả năng hoạt động của IoT -- AI -- Web -- App trong một hệ thống tích hợp, làm cơ sở phát triển hoàn thiện trong tương lai.

