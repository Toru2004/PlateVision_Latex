\section{Tính cấp thiết của đề tài}
Trong bối cảnh đô thị hóa nhanh chóng và sự gia tăng dân số tại các thành phố lớn, nhu cầu về một hệ thống giao thông hiệu quả và thông minh ngày càng trở nên cấp bách. Sự gia tăng mật độ dân số dẫn đến việc lưu thông phức tạp hơn, gây ra ùn tắc và kéo theo nhiều vấn đề như ô nhiễm môi trường, tai nạn giao thông và sự lãng phí thời gian. Bãi đỗ xe thông minh đã trở thành một trong những giải pháp quan trọng trong việc cải thiện tình hình này, không chỉ giúp tối ưu hóa lưu lượng giao thông trong các bãi giữ xe mà còn nâng cao trải nghiệm cho người sử dụng.

Bãi đỗ xe thông minh hoạt động dựa trên các công nghệ tiên tiến, đặc biệt là công nghệ nhận diện bằng camera và Machine learning, cho phép quét biển số xe một cách tự động. Điều này giúp việc quản lý và kiểm soát ra vào trở nên linh hoạt và hiệu quả.

Ngoài việc áp dụng trong bãi đỗ xe thông minh với trạm ra vào tự động còn có thể được triển khai tại nhiều lĩnh vực khác nhau, bao gồm khu vực ra vào trạm thu phí, khu dân cư, sân bay và các khu công nghiệp. Việc này không chỉ giúp giảm thiểu ùn tắc mà còn nâng cao tính an toàn và bảo mật cho các khu vực này. Đồng thời đảm bảo tài sản cá nhân người an toàn hơn trách các tình trạng mất cắp.

Bài nghiên cứu của nhóm sẽ phân tích thực trạng hiện nay của bãi đỗ xe và từ đó đề ra phương pháp. Bên cạnh đó, nhóm sẽ khám phá các khía cạnh quan trọng của trạm ra vào tự động, từ công nghệ ứng dụng, lợi ích mang lại cho người dùng và người quản lý, cho đến những thách thức trong quá trình triển khai. Qua đó, nhóm hy vọng sẽ cung cấp cái nhìn toàn diện về vai trò và tầm quan trọng của bãi đỗ xe thông minh trong hệ thống giao thông thông minh, góp phần xây dựng một môi trường giao thông an toàn và hiệu quả hơn cho tương lai.
