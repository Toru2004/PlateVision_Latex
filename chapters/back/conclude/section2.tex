\section{Hướng phát triển trong tương lai}

\subsection{Nâng cao độ chính xác và khả năng nhận diện}

Trong tương lai, hệ thống có thể được cải tiến bằng cách nghiên cứu và áp dụng
các kiến trúc học sâu tiên tiến như EfficientNet hoặc Vision Transformer nhằm
nâng cao độ chính xác nhận diện trong các điều kiện phức tạp như ánh sáng yếu,
góc chụp không thuận lợi hoặc biển số bị che khuất.

Bên cạnh đó, cơ chế học liên tục (Continual Learning) có thể được tích hợp để
cho phép mô hình tự động cập nhật và cải thiện hiệu năng dựa trên dữ liệu thu
thập thực tế trong quá trình vận hành, từ đó giúp hệ thống thích nghi tốt hơn
với sự thay đổi của môi trường và đặc điểm phương tiện.

\subsection{Mở rộng phạm vi hỗ trợ và đối tượng sử dụng}

Hệ thống có thể được mở rộng để hỗ trợ nhiều loại phương tiện khác nhau như
xe tải, xe buýt và xe điện, đáp ứng nhu cầu quản lý đa dạng trong các bãi đỗ
quy mô lớn. Đồng thời, việc phát triển giao diện đa ngôn ngữ sẽ giúp hệ thống
phù hợp hơn với người dùng quốc tế và các môi trường triển khai khác nhau.

Ngoài ra, các phương thức xác thực đa yếu tố như thẻ nhân viên hoặc căn cước
công dân (CCCD) gắn chip có thể được tích hợp nhằm tăng cường mức độ an toàn
và giảm thiểu rủi ro gian lận trong quá trình kiểm soát ra/vào.

\subsection{Tối ưu hóa hạ tầng và hiệu năng hệ thống}

Một hướng phát triển quan trọng là nghiên cứu áp dụng điện toán biên (Edge
Computing), cho phép xử lý dữ liệu trực tiếp tại thiết bị IoT nhằm giảm độ trễ
truyền thông và giảm tải cho máy chủ trung tâm. Các mô hình trí tuệ nhân tạo
có thể được tinh chỉnh, rút gọn và tối ưu để phù hợp với tài nguyên phần cứng
hạn chế của các thiết bị edge AI.

Bên cạnh đó, hệ thống có thể được cải tiến thông qua việc sử dụng các giao
thức truyền thông nhẹ như MQTT-SN hoặc CoAP nhằm tối ưu hóa băng thông và
nâng cao độ tin cậy trong quá trình trao đổi dữ liệu. Cơ chế vận hành theo
hướng \textit{offline-first} cũng là một giải pháp tiềm năng, giúp hệ thống
duy trì hoạt động ổn định khi mất kết nối Internet và tự động đồng bộ dữ liệu
khi kết nối được khôi phục.

\subsection{Mở rộng khả năng tích hợp và quy mô triển khai}

Trong tương lai, hệ thống có thể được phát triển theo hướng xây dựng một hệ
sinh thái mở thông qua các API chuẩn, cho phép tích hợp với các hệ thống quản
lý giao thông đô thị và tòa nhà thông minh. Ngoài ra, công nghệ Blockchain có
thể được nghiên cứu áp dụng trong quản lý giao dịch và vé xe nhằm đảm bảo
tính minh bạch, toàn vẹn và an toàn dữ liệu.

Hệ thống cũng có khả năng mở rộng để quản lý tập trung nhiều bãi đỗ xe
(Multi-parking management), từ quy mô nhỏ trong khuôn viên trường đại học
đến các bãi xe công cộng lớn như sân bay, trung tâm thương mại hoặc khu đô thị
thông minh.

\subsection{Định hướng triển khai AIoT trong tương lai}

Trong tương lai, hệ thống có thể được triển khai trên các thiết bị edge AI hoặc
phần cứng công nghiệp chuyên dụng, trong đó các mô hình trí tuệ nhân tạo sẽ
được tinh chỉnh và tối ưu để chạy trực tiếp trên thiết bị IoT. Cách tiếp cận
này giúp giảm độ trễ xử lý, tăng tính tự chủ của hệ thống và nâng cao khả năng
vận hành độc lập trong các môi trường triển khai thực tế quy mô lớn.
