\section{Hướng phát triển trong tương lai}

\subsection{Nâng cao độ chính xác và khả năng nhận diện}

Nâng cấp mô hình AI: Nghiên cứu ứng dụng các kiến trúc mới như EfficientNet, Vision Transformer nhằm tăng cường độ chính xác khi nhận diện trong điều kiện ánh sáng yếu hoặc biển số bị che mờ. 
Cơ chế tự học (Continual Learning): Phát triển khả năng tự cập nhật và cải thiện mô hình dựa trên dữ liệu thu thập thực tế theo thời gian.

\subsection{Mở rộng phạm vi hỗ trợ và đối tượng sử dụng}

Đa dạng phương tiện: Mở rộng khả năng kiểm soát cho các loại xe tải, xe buýt và xe điện nhằm đáp ứng nhu cầu đa dạng của người dùng. Hỗ trợ đa ngôn ngữ: Phát triển giao diện và hệ thống thông báo hỗ trợ nhiều ngôn ngữ để phục vụ người dùng quốc tế.
Xác thực đa phương thức: Tích hợp nhận diện thẻ nhân viên và căn cước công dân (CCCD) gắn chip để tăng cường tính bảo mật.

\subsection{Tối ưu hóa hạ tầng và hiệu năng hệ thống}

Điện toán biên (Edge Computing): Chuyển dịch xử lý hình ảnh trực tiếp tại thiết bị IoT để giảm độ trễ và giảm tải cho máy chủ trung tâm.
Cải tiến giao thức truyền thông: Nghiên cứu và áp dụng các giao thức mới như MQTT-SN hoặc CoAP để tối ưu hóa băng thông và độ tin cậy trong truyền dữ liệu giữa các thiết bị IoT và máy chủ.
Cơ chế Offline-first: Đảm bảo hệ thống duy trì vận hành ổn định ngay cả khi mất kết nối Internet và tự động đồng bộ hóa khi có mạng trở lại. 

\subsection{Mở rộng khả năng tích hợp và quy mô triển khai}

Hệ sinh thái thông minh: Phát triển API mở để kết nối với hệ thống quản lý giao thông đô thị và tòa nhà thông minh.
Công nghệ Blockchain: Ứng dụng Blockchain trong quản lý giao dịch và vé xe để đảm bảo tính minh bạch và bảo mật tuyệt đối.
Quy mô triển khai: Phát triển hệ thống quản lý tập trung đa bãi xe (Multi-parking management) từ quy mô nhỏ tới các trạm bãi công cộng lớn như sân bay, trung tâm thương mại.
