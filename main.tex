% File: main.tex
% Ho Chi Minh University of Transport Thesis template

\documentclass[a4paper,12pt]{report}

% =========================
% LOAD PACKAGES FIRST
% =========================
\usepackage[utf8]{inputenc}
\usepackage[utf8]{vietnam}
\usepackage{times}  % Font Times New Roman
\usepackage{graphicx}
\usepackage{geometry}
\usepackage{fancybox}
\usepackage{xcolor}
\usepackage{anyfontsize}
\usepackage{amsmath}
\usepackage{amssymb}
\usepackage{amsfonts}
\usepackage{enumitem}
\usepackage{float}
\usepackage{tikz}
\usepackage{placeins}
\usepackage{booktabs}
\usepackage{titlesec}
\usepackage[acronym]{glossaries}
\usepackage{tocloft}
\usepackage{setspace} 
\onehalfspacing
\usetikzlibrary{arrows.meta, positioning, shapes.geometric}

% =========================
% GEOMETRY
% =========================
\geometry{
    a4paper,
    left=3cm,
    right=2cm,
    top=2cm,
    bottom=2cm
}

% =========================
% FORMAT CHAPTER, SECTION, SUBSECTION
% =========================

% Format CHAPTER - 14pt, in hoa, in đậm, căn giữa
\titleformat{\chapter}[hang]
{\normalfont\fontsize{14}{17}\bfseries\centering}
{\MakeUppercase{\chaptertitlename\ \thechapter.}}
{1em}
{\MakeUppercase}

\titlespacing*{\chapter}{0pt}{0pt}{20pt}


% CHAPTER không số (chapter*)
\titleformat{name=\chapter,numberless}[hang]
{\normalfont\fontsize{14}{17}\bfseries\centering}
{}
{0pt}
{\MakeUppercase}

\titlespacing*{\chapter}{0pt}{0pt}{20pt}
% Format SECTION - 13pt, in đậm
\titleformat{\section}
{\normalfont\fontsize{13}{16}\bfseries}
{\thesection}
{1em}
{}

% Format SUBSECTION - 13pt, in đậm, in nghiêng
\titleformat{\subsection}
{\normalfont\fontsize{13}{16}\bfseries\itshape}
{\thesubsection}
{1em}
{}
%định dạng mục lục
% \widowpenalty=10000
% \clubpenalty=10000
% \raggedbottom
% =========================
% GLOSSARIES
% =========================
\makeglossaries

% =========================
% BEGIN DOCUMENT
% =========================
\begin{document}

% ===== TRANG BÌA - KHÔNG ĐÁNH SỐ =====
\pagenumbering{gobble}

% Trang 1
\begin{center}
    \doublebox{
    \begin{minipage}{0.9\textwidth}
        \vspace{0.5cm}
        \begin{center}
            {\fontsize{13}{15}\selectfont \textbf{TRƯỜNG ĐẠI HỌC GIAO THÔNG VẬN TẢI TP. HỒ CHÍ MINH}}
            
            \vspace{1cm}
            
            \includegraphics[width=0.6\textwidth]{graphics/front/logo_uth.png}
            
            \vspace{2cm}
            
            {\fontsize{22}{26}\selectfont \textbf{BÁO CÁO TỔNG KẾT}}
            
            \vspace{0.8cm}
            
            {\fontsize{16}{19}\selectfont \textbf{HỘI THẢO SINH VIÊN UTH}}
            
            {\fontsize{16}{19}\selectfont \textbf{NGHIÊN CỨU KHOA HỌC NĂM 2025}}
            
            \vspace{3cm}
            
            {\fontsize{14}{17}\selectfont \textbf{TÊN ĐỀ TÀI: NGHIÊN CỨU VÀ ỨNG DỤNG DEEP LEARNING KẾT HỢP IOT TRONG XÂY DỰNG HỆ THỐNG BÃI ĐỖ XE THÔNG MINH TRƯỜNG ĐẠI HỌC GIAO THÔNG VẬN TẢI TP.HCM}}
            
            \vspace{0.3cm}
            
            {\fontsize{14}{17}\selectfont \textbf{Mã số: ………………….}}
            
            \vspace{4cm}
            
            \begin{flushleft}
                {\fontsize{13}{15}\selectfont \textbf{Nhóm trưởng đề tài \quad : Lê Tuấn Khang}}
                
                \vspace{0.3cm}
                
                {\fontsize{13}{15}\selectfont \textbf{Thời gian thực hiện \quad : Từ tháng 9/2025 đến tháng 12/2025}}
            \end{flushleft}
            
            \vspace{2cm}
            
            {\fontsize{13}{15}\selectfont \textbf{Thành phố Hồ Chí Minh, tháng 12 năm 2025}}
            
            \vspace{0.5cm}
        \end{center}
    \end{minipage}
    }
\end{center}

\newpage

% Trang 2
\begin{center}
    \doublebox{
    \begin{minipage}{1.0\textwidth}
        \vspace{0.5cm}
        \begin{center}
            {\fontsize{13}{15}\selectfont \textbf{TRƯỜNG ĐẠI HỌC GIAO THÔNG VẬN TẢI TP. HỒ CHÍ MINH}}
            
            \vspace{1cm}
            
            \includegraphics[width=0.6\textwidth]{graphics/front/logo_uth.png}
            
            \vspace{1.0cm}
            
            {\fontsize{22}{26}\selectfont \textbf{BÁO CÁO TỔNG KẾT}}
            
            \vspace{0.8cm}
            
            {\fontsize{16}{19}\selectfont \textbf{HỘI THẢO SINH VIÊN UTH}}
            
            {\fontsize{16}{19}\selectfont \textbf{NGHIÊN CỨU KHOA HỌC NĂM 2025}}
            
            \vspace{2cm}
            
            {\fontsize{14}{17}\selectfont \textbf{ TÊN ĐỀ TÀI: NGHIÊN CỨU VÀ ỨNG DỤNG DEEP LEARNING KẾT HỢP IOT TRONG XÂY DỰNG HỆ THỐNG BÃI ĐỖ XE THÔNG MINH TRƯỜNG ĐẠI HỌC GIAO THÔNG VẬN TẢI TP.HCM}}
            
            \vspace{0.3cm}
            
            \begin{tabular}{ll}
                {\fontsize{14}{17}\selectfont \textbf{Mã số}} & {\fontsize{14}{17}\selectfont \textbf{:………………….}}
            \end{tabular}
            
            \vspace{1.0cm}
            
            \begin{flushleft}
                \hspace{1cm}{\fontsize{13}{15}\selectfont \textbf{\underline{DANH SÁCH THÀNH VIÊN:}}}
                
                \vspace{0.5cm}
                
                \hspace{1cm}{\fontsize{13}{15}\selectfont \textbf{1. Nhóm trưởng: Lê Tuấn Khang -- MSSV: 2251120420}}
                
                \vspace{0.3cm}
                
                \hspace{1cm}{\fontsize{13}{15}\selectfont \textbf{2. Thành viên: Hồ Huỳnh Nhu -- MSSV: 2251120433}}
                
                \vspace{0.3cm}
                
                \hspace{1cm}{\fontsize{13}{15}\selectfont \textbf{3. Thành viên: Nguyễn Thành Đạt -- MSSV: 2251120413}}
                \vspace{0.3cm}
                
                \hspace{1cm}{\fontsize{13}{15}\selectfont \textbf{4. Thành viên: Nguyễn Hồng Minh -- MSSV: 2251120428}}
                \vspace{0.3cm}
                
                \hspace{1cm}{\fontsize{13}{15}\selectfont \textbf{5. Thành viên: Lê Nguyễn Minh Phúc -- MSSV: 2251120040}}
            \end{flushleft}
            
            \vspace{0.8cm}
            
            \begin{flushleft}
                \hspace{1cm}{\fontsize{13}{15}\selectfont \textbf{\underline{GIẢNG VIÊN HƯỚNG DẪN:}}}
                
                \vspace{0.3cm}
                
                \hspace{1cm}{\fontsize{13}{15}\selectfont Họ và tên, học hàm, học vị: \textbf{ThS. Nguyễn Văn Huy}}
            \end{flushleft}
            
            \vspace{0.8cm}
            
            {\fontsize{13}{15}\selectfont \textbf{Thành phố Hồ Chí Minh, tháng 12 năm 2025}}
            \vspace{0.3cm}
          
        \end{center}
    \end{minipage}
    }
\end{center}

\clearpage

% ===== MỤC LỤC - ĐÁNH SỐ LA MÃ (i, ii, iii...) =====
% ===== MỤC LỤC - ĐÁNH SỐ LA MÃ (i, ii, iii...) =====

\chapter*{LỜI CÁM ƠN}


Nhóm nghiên cứu xin gửi lời cảm ơn chân thành đến:

ThS. Nguyễn Văn Huy, giảng viên hướng dẫn, đã tận tình chỉ bảo, định hướng và hỗ trợ nhóm trong suốt quá trình thực hiện đề tài. Sự hướng dẫn chi tiết và những góp ý quý báu của thầy đã giúp nhóm hoàn thiện nghiên cứu này.

Viện Công Nghệ Thông Tin và Điện, Điện tử, Trường Đại học Giao thông Vận tải TP. Hồ Chí Minh đã tạo điều kiện thuận lợi về cơ sở vật chất, trang thiết bị và môi trường học tập để nhóm có thể thực hiện đề tài.

Gia đình, bạn bè đã luôn động viên, khích lệ và tạo mọi điều kiện tốt nhất để các thành viên trong nhóm có thể tập trung hoàn thành nghiên cứu.

Mặc dù đã cố gắng hết sức, nhưng do thời gian và kinh nghiệm còn hạn chế, đề tài chắc chắn còn nhiều thiếu sót. Nhóm nghiên cứu rất mong nhận được sự góp ý, chỉ bảo của quý thầy cô và bạn đọc để đề tài được hoàn thiện hơn.

Nhóm nghiên cứu xin chân thành cảm ơn!

\vspace{2cm}

\begin{flushright}
\begin{minipage}{0.5\textwidth}
    \centering
    \textit{TP. Hồ Chí Minh, ngày 21 tháng 12 năm 2025}\\[0.3cm]
    \textbf{Đại diện nhóm thực hiện}\\[1.5cm]
   
    \textbf{Lê Tuấn Khang}
\end{minipage}
\end{flushright}

\clearpage


\chapter*{LỜI CAM ĐOAN}


Nhóm nghiên cứu xin cam đoan:

Đề tài "Bãi đỗ xe thông minh" là công trình nghiên cứu của nhóm dưới sự hướng dẫn của ThS. Nguyễn Văn Huy.

Các số liệu, kết quả nêu trong báo cáo là trung thực, có nguồn gốc rõ ràng và được trích dẫn đầy đủ theo quy định.

Những kết luận khoa học của đề tài chưa từng được ai công bố trong bất kỳ công trình nào khác.

Nếu phát hiện có bất kỳ gian lận nào, nhóm nghiên cứu xin hoàn toàn chịu trách nhiệm về nội dung đề tài của mình. Trường Đại học Giao thông Vận tải TP. Hồ Chí Minh không liên quan đến những vi phạm tác quyền, bản quyền do nhóm nghiên cứu gây ra trong quá trình thực hiện (nếu có).

\vspace{2cm}

\begin{flushright}
\begin{minipage}{0.5\textwidth}
    \centering
    \textit{TP. Hồ Chí Minh, ngày 21 tháng 12 năm 2025}\\[0.3cm]
    \textbf{Đại diện nhóm thực hiện}\\[1.5cm]
   
   
    \textbf{Lê Tuấn Khang}
\end{minipage}
\end{flushright}

\clearpage
\pagenumbering{roman}
\setcounter{page}{1}
% Tạm tắt format của tocloft
\renewcommand{\cfttoctitlefont}{\hspace*{\fill}\fontsize{14}{17}\selectfont\bfseries\MakeUppercase}
\renewcommand{\cftaftertoctitle}{\hspace*{\fill}}

\renewcommand{\cftchappresnum}{CHƯƠNG~}
\renewcommand{\cftchapaftersnum}{:}
\renewcommand{\cftchapfont}{\bfseries}
\renewcommand{\cftchappagefont}{\bfseries}
\setlength{\cftchapnumwidth}{6em}

\setlength{\cftaftertoctitleskip}{5pt}
\setlength{\cftbeforechapskip}{10pt}
\setlength{\cftbeforesecskip}{3pt}
\setlength{\cftbeforesubsecskip}{2pt}

\tableofcontents
\clearpage

\addcontentsline{toc}{chapter}{DANH MỤC HÌNH ẢNH}  % Thêm vào mục lục

% Format tiêu đề Danh mục hình
\renewcommand{\cftloftitlefont}{\hspace*{\fill}\fontsize{14}{17}\selectfont\bfseries}
\renewcommand{\cftafterloftitle}{\hspace*{\fill}}
\renewcommand{\listfigurename}{DANH MỤC HÌNH ẢNH}

% Format cho figure trong danh mục
\renewcommand{\cftfigfont}{\normalfont}
\renewcommand{\cftfigpagefont}{\normalfont}
\renewcommand{\cftfigpresnum}{Hình~}
\renewcommand{\cftfigaftersnum}{:~}
\setlength{\cftfignumwidth}{5em}
\setlength{\cftfigindent}{0pt}

% Điều chỉnh khoảng cách
\setlength{\cftbeforeloftitleskip}{-20pt}
\setlength{\cftafterloftitleskip}{10pt}
\setlength{\cftbeforefigskip}{5pt}

\listoffigures
\clearpage

% ===== DANH MỤC BẢNG =====
\addcontentsline{toc}{chapter}{DANH MỤC BẢNG}  % Thêm vào mục lục

% Format tiêu đề Danh mục bảng
\renewcommand{\cftlottitlefont}{\hspace*{\fill}\fontsize{14}{17}\selectfont\bfseries}
\renewcommand{\cftafterlottitle}{\hspace*{\fill}}
\renewcommand{\listtablename}{DANH MỤC BẢNG}

% Format cho table trong danh mục
\renewcommand{\cfttabfont}{\normalfont}
\renewcommand{\cfttabpagefont}{\normalfont}
\renewcommand{\cfttabpresnum}{Bảng~}
\renewcommand{\cfttabaftersnum}{:~}
% ===== BỎ KHOẢNG CÁCH GIỮA CÁC CHƯƠNG TRONG DANH MỤC BẢNG =====
\setlength{\cftbeforetabskip}{0pt}
\makeatletter

\makeatother

\setlength{\cfttabnumwidth}{5em}
\setlength{\cfttabindent}{0pt}


% Điều chỉnh khoảng cách
\setlength{\cftbeforelottitleskip}{-20pt}
\setlength{\cftafterlottitleskip}{10pt}
\setlength{\cftbeforetabskip}{5pt}


\listoftables


\clearpage
\addcontentsline{toc}{chapter}{DANH MỤC TỪ VIẾT TẮT}  % Thêm vào mục lục
\clearpage
\begin{center}
\textbf{\MakeUppercase{DANH MỤC TỪ VIẾT TẮT}}
\end{center}


\begin{center}
\begin{longtable}{|p{4cm}|p{9cm}|}
\hline
\textbf{Từ viết tắt} & \textbf{Diễn giải} \\ \hline
AI/ML & Artificial Intelligence / Machine Learning \\ \hline

OCR & Optical Character Recognition \\ \hline
YOLOv8 & You Only Look Once version 8 \\ \hline
CNN & Convolutional Neural Network \\ \hline
ReLU & Rectified Linear Unit \\ \hline
UI & User Interface \\ \hline
CRUD & Create, Read, Update, Delete \\ \hline
MVVM & Model-View-ViewModel \\ \hline
GPU & Graphics Processing Unit \\ \hline
LFW & Labeled Faces in the Wild \\ \hline
KNN & K-Nearest Neighbors \\ \hline
SVM & Support Vector Machine \\ \hline
HOG & Histogram of Oriented Gradients \\ \hline
mAP & mean Average Precision \\ \hline
IoU & Intersection over Union \\ \hline
OCR & Optical Character Recognition \\ \hline
SSD & Single Shot MultiBox Detector \\ \hline
API & Application Programming Interface \\ \hline
\end{longtable}
\end{center}

\clearpage


\clearpage
% Bỏ comment nếu cần
% \listoffigures
% \clearpage
% \listoftables
% \clearpage

% ===== NỘI DUNG CHÍNH - ĐÁNH SỐ Ả-RẬP (1, 2, 3...) =====
\pagenumbering{arabic}
\setcounter{page}{1}

% Các chapter
\chapter*{\centering\Large{LỜI MỞ ĐẦU}}
\addcontentsline{toc}{chapter}{LỜI MỞ ĐẦU}

% Renumber sections to show only the section number (1, 2, 3...) instead of chapter.section (0.1, 0.2, 0.3...)
\renewcommand{\thesection}{\arabic{section}}
\setcounter{section}{0}

\section{Tính cấp thiết của đề tài}
Trong bối cảnh đô thị hóa nhanh chóng và sự gia tăng dân số tại các thành phố lớn, nhu cầu về một hệ thống giao thông hiệu quả và thông minh ngày càng trở nên cấp bách. Sự gia tăng mật độ dân số dẫn đến việc lưu thông phức tạp hơn, gây ra ùn tắc và kéo theo nhiều vấn đề như ô nhiễm môi trường, tai nạn giao thông và sự lãng phí thời gian. Bãi đỗ xe thông minh đã trở thành một trong những giải pháp quan trọng trong việc cải thiện tình hình này, không chỉ giúp tối ưu hóa lưu lượng giao thông trong các bãi giữ xe mà còn nâng cao trải nghiệm cho người sử dụng.

Bãi đỗ xe thông minh hoạt động dựa trên các công nghệ tiên tiến, đặc biệt là công nghệ nhận diện bằng camera và Machine learning, cho phép quét biển số xe một cách tự động. Điều này giúp việc quản lý và kiểm soát ra vào trở nên linh hoạt và hiệu quả.

Ngoài việc áp dụng trong bãi đỗ xe thông minh với trạm ra vào tự động còn có thể được triển khai tại nhiều lĩnh vực khác nhau, bao gồm khu vực ra vào trạm thu phí, khu dân cư, sân bay và các khu công nghiệp. Việc này không chỉ giúp giảm thiểu ùn tắc mà còn nâng cao tính an toàn và bảo mật cho các khu vực này. Đồng thời đảm bảo tài sản cá nhân người an toàn hơn trách các tình trạng mất cắp.

Bài nghiên cứu của nhóm sẽ phân tích thực trạng hiện nay của bãi đỗ xe và từ đó đề ra phương pháp. Bên cạnh đó, nhóm sẽ khám phá các khía cạnh quan trọng của trạm ra vào tự động, từ công nghệ ứng dụng, lợi ích mang lại cho người dùng và người quản lý, cho đến những thách thức trong quá trình triển khai. Qua đó, nhóm hy vọng sẽ cung cấp cái nhìn toàn diện về vai trò và tầm quan trọng của bãi đỗ xe thông minh trong hệ thống giao thông thông minh, góp phần xây dựng một môi trường giao thông an toàn và hiệu quả hơn cho tương lai.

\section{Phạm vi nghiên cứu}

Đề tài ``Bãi đỗ xe thông minh'' tập trung nghiên cứu và xây dựng mô hình hệ thống quản lý bãi đỗ xe tự động, hỗ trợ giám sát -- đặt chỗ -- nhận diện biển số phương tiện. Phạm vi nghiên cứu của đề tài bao gồm các nội dung chính sau:

\begin{itemize}
    \item Hệ thống phần cứng IoT tại bãi đỗ xe:
    \begin{itemize}
        \item Thiết kế mô hình bãi đỗ xe thu nhỏ hoặc hệ thống thử nghiệm.
        \item Sử dụng các cảm biến siêu âm, RFID/thẻ từ, camera để phát hiện phương tiện và trạng thái chỗ đỗ.
        \item Kết nối thiết bị với vi điều khiển (ESP32/Arduino hoặc tương đương) để truyền dữ liệu lên máy chủ.
    \end{itemize}

    \item Nền tảng quản lý tập trung (Web Admin):
    \begin{itemize}
        \item Theo dõi tình trạng bãi xe theo thời gian thực.
        \item Thêm/xóa/chỉnh sửa thông tin xe, người dùng.
        \item Xem báo cáo thống kê doanh thu.
        \item Hệ thống backend cung cấp API xử lý dữ liệu từ IoT và ứng dụng người dùng.
    \end{itemize}

    \item Ứng dụng di động cho người dùng:
    \begin{itemize}
        \item Xem thông tin tài khoản, nội dung quy định của hệ thống.
        \item Xem trạng thái, check-in/check-out phương tiện.
        \item Xem lịch sử gửi xe, thông báo trạng thái.
    \end{itemize}

    \item Hệ thống AI nhận diện biển số xe (LPR -- License Plate Recognition):
    \begin{itemize}
        \item Triển khai mô hình nhận diện biển số bằng YOLO + CRNN.
        \item Tự động ghi nhận dữ liệu biển số và đối chiếu thông tin với hệ thống quản lý.
    \end{itemize}

    \item Giới hạn phạm vi của đề tài:
    \begin{itemize}
        \item Hệ thống được xây dựng ở mức prototype, quy mô thử nghiệm.
        \item Chưa xử lý nâng cao như biển số mờ, điều kiện ánh sáng kém, quy mô lớn ngoài thực tế.
        \item Tính năng thanh toán trực tuyến mới dừng lại ở mức mô phỏng tích hợp cơ bản.
        \item Hệ thống bảo mật ở mức cơ bản, chưa đạt tiêu chuẩn thương mại hóa.
    \end{itemize}
\end{itemize}

Phạm vi nghiên cứu tập trung vào việc mô hình hóa hệ thống, xây dựng phiên bản thử nghiệm khả thi, đánh giá khả năng hoạt động của IoT -- AI -- Web -- App trong một hệ thống tích hợp, làm cơ sở phát triển hoàn thiện trong tương lai.


\section{Phương pháp nghiên cứu}

\subsection{Phương pháp khảo sát thực tiễn}

Để thu thập thông tin phục vụ cho việc đánh giá thực trạng sử dụng bãi giữ xe tại trường, nhóm tiến hành khảo sát thực tiễn đối với sinh viên thường xuyên gửi xe trong khuôn viên trường. Tổng cộng \textbf{151 sinh viên} đã tham gia khảo sát.

Khảo sát được thực hiện theo hình thức trực tuyến thông qua Google Forms với các nội dung như sau:

\begin{itemize}
    \item Thói quen sử dụng bãi giữ xe hiện tại
    \item Mức độ gặp sự cố khi sử dụng thẻ giữ xe truyền thống
    \item Mức độ bất tiện mà người dùng cảm nhận
    \item Nhu cầu của người dùng đối với các giải pháp thay thế như giữ xe không cần thẻ
\end{itemize}

Dữ liệu sau khi thu thập được dùng làm cơ sở đánh giá thực trạng và xác định các vấn đề tồn tại trong quy trình giữ xe của sinh viên.

\subsection{Phương pháp phân tích - tổng hợp}

\subsubsection{Phương pháp thống kê mô tả}

Dữ liệu thu thập từ Google Forms được xuất sang định dạng Excel để xử lý. Nhóm sử dụng phần mềm Microsoft Excel để:

\begin{itemize}
    \item Tính toán tỷ lệ phần trăm các câu trả lời
    \item Lập biểu đồ tròn, biểu đồ cột để trực quan hóa dữ liệu
    \item Xác định các vấn đề phổ biến nhất mà người dùng gặp phải
\end{itemize}

\subsubsection{Phương pháp so sánh và đối chiếu}

Kết quả khảo sát được so sánh với các nghiên cứu tương tự về hệ thống quản lý bãi giữ xe thông minh đã công bố, nhằm đánh giá mức độ phổ biến của vấn đề và tính cấp thiết của giải pháp.

\subsubsection{Phương pháp tổng hợp}

Từ dữ liệu khảo sát và tài liệu nghiên cứu, nhóm tổng hợp để:

\begin{itemize}
    \item Xác định các vấn đề tồn tại trong hệ thống giữ xe hiện tại
    \item Làm rõ nhu cầu và kỳ vọng của người dùng
    \item Đề xuất các yêu cầu chức năng cho hệ thống mới
\end{itemize}

\subsection{Phương pháp mô phỏng}

Để kiểm chứng tính khả thi của giải pháp đề xuất, nhóm tiến hành xây dựng mô hình thử nghiệm bao gồm ba thành phần chính:


\subsubsection{Ứng dụng Desktop App}

Hệ thống nhận diện biển số xe được phát triển dưới dạng ứng dụng desktop, sử dụng các thuật toán của Deep Learning. Cụ thể là CNN,YOLO, Siamese, Deepface để tự động xử lý dữ liệu xe khi vào/ra bãi. Ứng dụng được cài đặt tại cổng ra vào bãi giữ xe, kết nối với camera để thu thập, xử lý hình ảnh thời gian thực và gửi lên cơ sở dữ liệu.

\subsubsection{Ứng dụng di động (Mobile App)}

Ứng dụng dành cho người dùng (sinh viên, bảo vệ) được phát triển trên nền tảng Android sử dụng Android Studio. Ứng dụng cho phép sinh viên đăng ký thông tin xe, theo dõi lịch sử ra vào bãi giữ xe và thanh toán phí trực tuyến. Ngoài ra ứng dụng cho phép bảo vệ xem lịch sử ra vào của từng xe, trạng thái cổng và có thể quét QR để yêu cầu mở cổng khi cần thiết.

\subsubsection{Trang web quản trị (Web Admin)}

Trang web quản trị được xây dựng để phục vụ nhân viên quản lý bãi xe, cho phép giám sát tình trạng bãi xe, quản lý thông tin sinh viên, bảo vệ, tra cứu lịch sử ra/vào, trạng thái bãi đỗ và thống kê doanh thu. 

\subsubsection{Quy trình thử nghiệm}

Hệ thống được thử nghiệm trên môi trường mô phỏng với các kịch bản sử dụng thực tế:

\begin{itemize}
    \item Kiểm tra độ chính xác nhận diện biển số xe trong các điều kiện ánh sáng khác nhau
    \item Đánh giá thời gian phản hồi của hệ thống từ lúc xe vào đến khi cổng mở/đóng
    \item Thử nghiệm đồng bộ dữ liệu giữa ba thành phần (Mobile App, Desktop App, Web Admin)
    \item Thu thập phản hồi từ người dùng thử nghiệm để cải tiến giao diện và tính năng
\end{itemize}

Kết quả thực nghiệm được ghi nhận và phân tích để đánh giá hiệu quả của giải pháp so với hệ thống giữ xe truyền thống.


\section{Cơ sở khoa học và ý nghĩa thực tiễn của đề tài}
\clearpage
\section{Các nội dung nghiên cứu chính của đề tài}

% Reset section numbering to default (Chapter.Section) for the rest of the document
\renewcommand{\thesection}{\thechapter.\arabic{section}}

\chapter{CƠ SỞ LÝ THUYẾT VÀ TỔNG QUAN TÀI LIỆU}
\label{chap:chap1-introduce}

\section{Cơ sở lý thuyết}

\subsection{Hệ thống nhận dạng hình ảnh (Computer Vision)}

Computer Vision là lĩnh vực cho phép máy tính hiểu và xử lý thông tin từ hình ảnh. Các bước xử lý cơ bản:

\textbf{Tiền xử lý ảnh:}
\begin{itemize}
    \item \textbf{Grayscale}: $I_{gray} = 0.299R + 0.587G + 0.114B$
    \item \textbf{Normalization}: $I_{norm} = \frac{I - \mu}{\sigma}$
    \item \textbf{Histogram Equalization}: Tăng độ tương phản
\end{itemize}

\textbf{Trích xuất đặc trưng:}
\begin{itemize}
    \item Edge detection (Sobel, Canny)
    \item Feature descriptors (SIFT, HOG)
    \item Deep features (CNN)
\end{itemize}

\textbf{Ứng dụng trong hệ thống:} Tiền xử lý ảnh từ camera, chuẩn hóa trước khi đưa vào các models AI.
% \subsection{Mạng nơ-ron tích chập (CNN)}

% Mạng nơ-ron tích chập (Convolutional Neural Network – CNN) là kiến trúc mạng học sâu được thiết kế để xử lý dữ liệu dạng lưới, đặc biệt là hình ảnh. CNN khai thác phép tích chập nhằm trích xuất các đặc trưng không gian quan trọng từ ảnh đầu vào.

% Kiến trúc CNN bao gồm ba thành phần chính:
% \begin{itemize}
%     \item \textbf{Lớp tích chập (Convolution)}: Áp dụng các bộ lọc để trích xuất đặc trưng cục bộ từ ảnh.
%     \item \textbf{Lớp gộp (Pooling)}: Giảm kích thước không gian của đặc trưng, giúp giảm nhiễu và chi phí tính toán.
%     \item \textbf{Lớp kết nối đầy đủ (Fully Connected)}: Thực hiện phân loại dựa trên các đặc trưng đã học.
% \end{itemize}

% Phép tích chập tại một vị trí $(i,j)$ được biểu diễn như sau:
% \begin{equation}
% y_{i,j} = \sum_{m=0}^{M-1} \sum_{n=0}^{N-1} x_{i+m, j+n} \cdot w_{m,n} + b
% \end{equation}

% Hàm kích hoạt ReLU thường được sử dụng để tăng tính phi tuyến:
% \begin{equation}
% \text{ReLU}(x) = \max(0, x)
% \end{equation}

% Quá trình huấn luyện CNN dựa trên thuật toán lan truyền ngược (backpropagation) kết hợp các thuật toán tối ưu như SGD hoặc Adam nhằm tối thiểu hóa hàm mất mát.
\subsection{Mạng nơ-ron tích chập (CNN)}

CNN là kiến trúc mạng neural chuyên xử lý dữ liệu dạng lưới như hình ảnh.
\subsubsection{Kiến trúc cơ bản}

\textbf{1. Convolutional Layer:}
\begin{equation}
Y_{i,j} = \sigma\left(\sum_{m}\sum_{n} W_{m,n} \cdot X_{i+m,j+n} + b\right)
\end{equation}

trong đó $X$ là ảnh đầu vào, $W$ là kernel (bộ lọc), $b$ là bias, 
$\sigma(\cdot)$ là hàm kích hoạt (ReLU), và $Y_{i,j}$ là giá trị đặc trưng tại vị trí $(i,j)$.

\textbf{2. Pooling Layer:}
\begin{equation}
Y_{i,j} = \max_{m,n \in R} X_{i+m,j+n}
\end{equation}

trong đó $R$ là vùng pooling (thường kích thước $2 \times 2$), 
và $Y_{i,j}$ là giá trị sau khi giảm kích thước không gian.

\textbf{3. Fully Connected Layer:}
\begin{equation}
y = \sigma(W \cdot x + b)
\end{equation}

trong đó $x$ là vector đặc trưng đầu vào, $W$ là ma trận trọng số, 
$b$ là bias và $y$ là đầu ra của lớp kết nối đầy đủ.

\textbf{4. Softmax (Output):}
\begin{equation}
P(y=k) = \frac{e^{z_k}}{\sum_{j=1}^{K} e^{z_j}}
\end{equation}

trong đó $z_k$ là giá trị logit của lớp $k$, $K$ là tổng số lớp,
và $P(y=k)$ là xác suất dự đoán mẫu thuộc về lớp $k$.


\subsubsection{Ứng dụng trong hệ thống}

\textbf{Nhiệm vụ:} Nhận dạng ký tự trên biển số xe

\textbf{Kiến trúc mô hình:}
\begin{center}
\begin{tabular}{l}
Đầu vào (32×32×1) \\
$\rightarrow$ Conv2D(32, 3×3) $\rightarrow$ ReLU $\rightarrow$ MaxPool(2×2) \\
$\rightarrow$ Conv2D(64, 3×3) $\rightarrow$ ReLU $\rightarrow$ MaxPool(2×2) \\
$\rightarrow$ Flatten $\rightarrow$ Dense(128) $\rightarrow$ Dropout(0.5) \\
$\rightarrow$ Dense(36) $\rightarrow$ Softmax
\end{tabular}
\end{center}

\textbf{Đầu ra:} 36 lớp ký tự (A–Z và 0–9)

\textbf{Quá trình huấn luyện:}
\begin{itemize}
    \item Hàm mất mát: Categorical Cross-Entropy
    \item Thuật toán tối ưu: Adam (learning rate = 0.001)
    \item Tập dữ liệu: hơn 100.000 ảnh ký tự, có áp dụng tăng cường dữ liệu (data augmentation)
\end{itemize}

\subsection{YOLO }
YOLO (You Only Look Once) là một thuật toán phát hiện đối tượng thời gian thực, trong đó toàn bộ ảnh đầu vào được xử lý chỉ trong một lần lan truyền xuôi (forward pass) của mạng neural. Nhờ cơ chế dự đoán đồng thời vị trí và nhãn đối tượng, YOLO cho phép đạt tốc độ xử lý cao nhưng vẫn đảm bảo độ chính xác. Hiệu năng của mô hình YOLO sau huấn luyện không được đánh giá qua hàm mất mát mà thông qua các chỉ số định lượng dựa trên sự so sánh giữa dự đoán và ground truth.

\paragraph{Đầu ra của mô hình YOLO}
Mô hình YOLO chia ảnh thành lưới $S \times S$. Mỗi ô lưới dự đoán $B$ bounding boxes. Mỗi box bao gồm tọa độ tâm $(x, y)$, chiều rộng và chiều cao $(w, h)$, cùng độ tin cậy (confidence score) $C$ thể hiện xác suất tồn tại đối tượng và độ chính xác của box đó:
\begin{equation}
C = \text{Pr(Object)} \times \text{IoU}_{\text{pred}}^{\text{truth}}
\end{equation}
Đồng thời, mỗi ô lưới dự đoán phân phối xác suất điều kiện cho các lớp: $\text{Pr(Class}i | \text{Object)}$. Điểm số cuối cùng cho mỗi bounding box và lớp được kết hợp như sau:
\begin{equation}
\text{Score}{\text{class}} = C \times \text{Pr(Class}_i | \text{Object)}
\end{equation}
Các dự đoán đầu ra sẽ được lọc thông qua Ngưỡng Độ Tin Cậy (Confidence Threshold) và thuật toán Ức chế Không cực đại (Non-Maximum Suppression - NMS) để loại bỏ các box trùng lặp, cho ra tập kết quả cuối cùng.

\subsubsection{Các chỉ số đánh giá hiệu năng phát hiện đối tượng}

\paragraph{Intersection over Union (IoU)}
IoU đo lường mức độ chồng lấp giữa khung bao dự đoán $B_{\text{pred}}$ và khung bao thực tế (ground truth) $B_{\text{gt}}$:
\begin{equation}
\text{IoU} = \frac{\text{Area}(B_{\text{pred}} \cap B_{\text{gt}})}{\text{Area}(B_{\text{pred}} \cup B_{\text{gt}})}
\end{equation}
Một dự đoán được coi là \textit{đúng} (True Positive) nếu IoU của nó với một ground truth lớn hơn một ngưỡng xác định (thường là 0.5) và phân loại đúng lớp.

\paragraph{Precision và Recall}
Dựa trên số lượng True Positive (TP), False Positive (FP) và False Negative (FN), ta tính:
\begin{align}
\text{Precision} &= \frac{TP}{TP + FP} \quad \text{(Độ chính xác)} \
\text{Recall} &= \frac{TP}{TP + FN} \quad \text{(Độ bao phủ)}
\end{align}
\begin{itemize}
\item \textbf{TP}: Dự đoán có IoU $\ge$ ngưỡng (vd: 0.5) và phân loại đúng.
\item \textbf{FP}: Dự đoán có IoU $\ge$ ngưỡng nhưng phân loại sai, hoặc IoU $<$ ngưỡng, hoặc là box dư thừa sau NMS.
\item \textbf{FN}: Đối tượng thật không được phát hiện bởi bất kỳ dự đoán đủ tốt nào.
\end{itemize}

\paragraph{Đường cong Precision--Recall và Average Precision (AP)}
Bằng cách thay đổi ngưỡng confidence, ta có thể vẽ đường cong Precision--Recall ($P = f(R)$). Average Precision (AP) cho một lớp là diện tích dưới đường cong này:
\begin{equation}
\text{AP} = \int_{0}^{1} P(R), dR \approx \sum_{k=1}^{n} P(k) \Delta R(k)
\end{equation}

\paragraph{Mean Average Precision (mAP)}
mAP là chỉ số tổng hợp quan trọng nhất, được tính bằng trung bình AP của tất cả các lớp.
\begin{itemize}
\item \textbf{mAP@50}: Sử dụng ngưỡng IoU cố định là 0.5.
\begin{equation}
\text{mAP@50} = \frac{1}{N} \sum_{c=1}^{N} \text{AP}c \quad \text{với } \text{IoU} \ge 0.5
\end{equation}
\item \textbf{mAP@50:95}: Chỉ số toàn diện hơn, tính trung bình mAP tại các ngưỡng IoU từ 0.5 đến 0.95 với bước nhảy 0.05.
\begin{equation}
\text{mAP@50:95} = \frac{1}{10} \sum{k=0}^{9} \text{mAP@}(0.5 + 0.05k)
\end{equation}
\end{itemize}

\paragraph{Nhận xét}
mAP@50:95 là thước đo nghiêm ngặt, đánh giá đồng thời khả năng phân loại và định vị chính xác của mô hình ở nhiều mức độ khắt khe khác nhau. Do đó, nó được xem là chỉ số chính để đánh giá hiệu năng tổng thể của các mô hình phát hiện đối tượng như YOLO. 
\subsubsection{Ứng dụng trong hệ thống}

Trong hệ thống đề xuất, ba mô hình YOLO được triển khai nhằm phục vụ các
nhiệm vụ nhận dạng khác nhau, bao gồm phát hiện biển số xe, nhận dạng ký tự
trên biển số và phát hiện logo phương tiện.

\textbf{1. Mô hình YOLO 1 – Phát hiện biển số xe:}
\begin{itemize}
    \item Đầu vào: Ảnh phương tiện đầy đủ, kích thước $640 \times 640$, tập dữ liệu gồm khoảng 4.500 ảnh;
    \item Số lớp: 1 lớp (license\_plate);
    \item Đầu ra: Tọa độ bounding box của biển số xe trong ảnh.
\end{itemize}

\textbf{2. Mô hình YOLO 2 – Phát hiện và phân tách ký tự:}
\begin{itemize}
    \item Đầu vào: Ảnh biển số đã được cắt (crop), kích thước $416 \times 416$, tập dữ liệu gồm khoảng 2.000 ảnh;
    \item Số lớp: 1 lớp (character\_area);
    \item Đầu ra: Vị trí và nhãn của từng ký tự trên biển số.
\end{itemize}

\textbf{3. Mô hình YOLO 3 – Phát hiện logo hãng xe:}
\begin{itemize}
    \item Đầu vào: Ảnh phần đầu xe ô tô, kích thước $640 \times 640$, tập dữ liệu gồm khoảng 18.000 ảnh;
    \item Số lớp: 20 lớp logo thương hiệu (Toyota, Honda, Mazda, \ldots);
    \item Đầu ra: Bounding box và nhãn thương hiệu logo tương ứng.
\end{itemize}

\textbf{Cấu hình huấn luyện và triển khai:}
\begin{itemize}
    \item Backbone: CSPDarknet53;
    \item Phiên bản mô hình: YOLOv8;
    \item Tốc độ xử lý: từ 60 đến 140 FPS khi triển khai trên GPU;
    \item Độ chính xác: mAP@0.5 đạt trên 90\%.
    

\end{itemize}



\subsection{Mạng Siamese Network}

Siamese Network là kiến trúc mạng nơ-ron gồm hai nhánh CNN chia sẻ trọng số, 
được thiết kế để học độ tương đồng giữa hai ảnh đầu vào thông qua khoảng cách 
trong không gian đặc trưng.

Hai ảnh đầu vào được đưa qua mạng CNN để trích xuất embedding $\mathbf{f}_1$ 
và $\mathbf{f}_2$. Khoảng cách Euclid giữa hai embedding được tính:
\begin{equation}
d(\mathbf{f}_1, \mathbf{f}_2) = \|\mathbf{f}_1 - \mathbf{f}_2\|_2
\end{equation}

Hàm mất mát Contrastive Loss được sử dụng để huấn luyện:
\begin{equation}
\mathcal{L}_{contrastive} = y \, d^2 + (1-y)\max(0, m - d)^2
\end{equation}

trong đó $y \in \{0,1\}$ biểu thị hai mẫu có cùng lớp hay không, 
$d$ là khoảng cách embedding, và $m$ là margin phân tách các cặp khác lớp.
\textbf{Ứng dụng vào hệ thống bãi đỗ xe thông minh:}
\begin{itemize}
    \item Mạng Siamese Network được huấn luyện trên tập dữ liệu VeRi-776, bao gồm khoảng 32.000 ảnh đầu xe, nhằm phục vụ việc so sánh và xác thực phương tiện khi ra vào bãi đỗ.
    
    \item Khi xe đi vào bãi, hệ thống camera chụp ảnh đầu xe và trích xuất vector đặc trưng (embedding) $\mathbf{f}_{in}$; embedding này được lưu trữ trong cơ sở dữ liệu.
    
    \item Khi xe rời khỏi bãi, hệ thống tiếp tục trích xuất embedding $\mathbf{f}_{out}$ từ ảnh đầu xe tại cổng ra và tiến hành so sánh với embedding $\mathbf{f}_{in}$ đã lưu.
    
    \item Nếu khoảng cách $d(\mathbf{f}_{in}, \mathbf{f}_{out}) < \tau$ (với $\tau$ là ngưỡng xác định trước), phương tiện được xác nhận là cùng một xe.
    
    \item Ngược lại, nếu khoảng cách lớn hơn hoặc bằng ngưỡng $\tau$, hệ thống sẽ phát hiện và cảnh báo hành vi bất thường.
\end{itemize}

\subsection{Công nghệ nhận dạng khuôn mặt (DeepFace)}

DeepFace là kiến trúc mạng nơ-ron tích chập (CNN) sâu được phát triển bởi 
Facebook, đạt độ chính xác 97.35\% trên bộ dữ liệu LFW, gần với khả năng 
nhận dạng của con người.

Các bước xử lý trong DeepFace:
\begin{itemize}
    \item \textbf{Phát hiện khuôn mặt}: Xác định vị trí khuôn mặt trong ảnh đầu vào.
    \item \textbf{Chuẩn hóa (Alignment)}: Căn chỉnh khuôn mặt về tư thế chuẩn 
    dựa trên các điểm đặc trưng.
    \item \textbf{Trích xuất đặc trưng}: Đưa ảnh qua mạng CNN để tạo vector 
    đặc trưng (embedding).
    \item \textbf{So khớp (Verification)}: Đánh giá độ tương đồng giữa các 
    vector đặc trưng.
\end{itemize}

Độ tương đồng giữa hai vector đặc trưng $\mathbf{f}_1$ và $\mathbf{f}_2$ 
được tính bằng độ đo cosine:
\begin{equation}
\text{sim}(\mathbf{f}_1, \mathbf{f}_2) =
\frac{\mathbf{f}_1 \cdot \mathbf{f}_2}
{\|\mathbf{f}_1\| \|\mathbf{f}_2\|}
\end{equation}

Giá trị $\text{sim}(\mathbf{f}_1, \mathbf{f}_2) \in [-1,1]$, giá trị càng 
lớn thể hiện độ tương đồng càng cao. Việc xác thực được thực hiện bằng cách 
so sánh giá trị này với ngưỡng xác định trước.\\
\textbf{Ứng dụng trong hệ thống bãi đỗ xe máy:}
\begin{itemize}
    \item DeepFace được tích hợp để xác thực khuôn mặt người lái xe máy, nhằm tăng cường bảo mật cho hệ thống.
    
    \item Khi xe máy vào bãi, camera chụp ảnh khuôn mặt người lái và trích xuất vector đặc trưng (embedding) $\mathbf{f}_{in}$, sau đó lưu cùng với thông tin biển số xe vào cơ sở dữ liệu.
    
    \item Khi xe ra bãi, hệ thống chụp lại khuôn mặt người lái hiện tại, trích xuất embedding $\mathbf{f}_{out}$ và tính độ tương đồng giữa hai embedding bằng công thức:
    
    \[
    \text{sim}(\mathbf{f}_{in}, \mathbf{f}_{out}) > \tau_{face}
    \]
    
    \item Nếu độ tương đồng lớn hơn ngưỡng $\tau_{face}$ (thường $\tau_{face} \geq 0.6$), người lái được xác nhận là chủ xe hợp lệ và cổng sẽ tự động mở.
    
    \item Ngược lại, hệ thống cảnh báo hành vi bất thường và gửi thông báo đến ứng dụng Mobile của chủ xe, giúp phát hiện các trường hợp xe bị trộm hoặc sử dụng trái phép.
\end{itemize}

\subsection{Các công nghệ và kiến trúc hệ thống đề xuất}

Hệ thống quản lý bãi đỗ xe thông minh được xây dựng dựa trên sự tích hợp
giữa các công nghệ Internet of Things (IoT), trí tuệ nhân tạo (AI) và các
nền tảng ứng dụng hiện đại, nhằm đảm bảo khả năng nhận dạng chính xác,
xử lý thời gian thực và vận hành ổn định trong môi trường thực tế.

\subsubsection{Vi điều khiển ESP32 và chức năng IoT trong hệ thống bãi đỗ xe}

ESP32 được sử dụng làm vi điều khiển trung tâm trong mô hình hệ thống IoT
mô phỏng, nhằm minh họa nguyên lý kết nối, điều khiển và giám sát thiết bị
trong bãi đỗ xe thông minh. Việc sử dụng ESP32 giúp hệ thống dễ triển khai
và phù hợp cho mục đích nghiên cứu.

Chức năng IoT trong mô hình hệ thống:
\begin{itemize}
    \item Mô phỏng việc nhận lệnh điều khiển theo thời gian thực từ hệ thống trung tâm;
    \item Mô phỏng điều khiển cổng ra vào thông qua servo motor;
    \item Mô phỏng việc thu thập dữ liệu từ các cảm biến để phát hiện xe và vật cản;
    \item Mô phỏng cơ chế cảnh báo khi phát hiện tác động bất thường lên cổng;
    \item Gửi trạng thái hoạt động của hệ thống về server để phục vụ giám sát.
\end{itemize}



\subsubsection{Ứng dụng di động với Kotlin Jetpack Compose}

Ứng dụng Android được phát triển bằng Kotlin và Jetpack Compose theo mô hình
UI khai báo (declarative), giúp đơn giản hóa việc quản lý giao diện và trạng thái.
Các chức năng chính:
\begin{itemize}
    \item Đăng nhập/đăng ký người dùng;
    \item Đăng ký thông tin cá nhân phương tiện;
    \item Thực hiện check-in/check-out;
    \item Xem lịch sử ra vào (bảo vệ) và nhận thông báo thời gian thực.
\end{itemize}

\subsubsection{Hệ thống web quản trị với Nuxt.js}
Hệ thống web quản trị được xây dựng trên nền tảng Nuxt.js, cung cấp giao diện
quản lý bãi đỗ xe trực quan và hiệu quả cho quản trị viên.
Chức năng chính của dashboard:
\begin{itemize}
    \item Theo dõi số lượng xe và tình trạng bãi đỗ;
    \item Quản lý phương tiện và người dùng;
    \item Xuất báo cáo thống kê;
    \item Cập nhật dữ liệu theo thời gian thực.
\end{itemize}

\subsubsection{Luồng xử lý và logic quyết định}

Quyết định mở cổng được đưa ra dựa trên kết quả nhận dạng và xác thực:

\begin{equation}
\text{OPEN\_GATE} =
\begin{cases}
\text{True}, & \text{if } (plate\_match \land auth\_match) \\
\text{False}, & \text{otherwise}
\end{cases}
\end{equation}

Trong đó:
\begin{itemize}
    \item $\text{OPEN\_GATE}$: Trạng thái điều khiển cổng (mở hoặc đóng);
    \item $\text{plate\_match}$: Kết quả so khớp biển số xe với cơ sở dữ liệu
    (True nếu trùng khớp);
    \item $\text{auth\_match}$: Kết quả xác thực người dùng hoặc phương tiện
    (ví dụ: khuôn mặt, phương tiện đã đăng ký);
    \item $\land$: Phép toán logic \textit{AND}.
\end{itemize}



Luồng xử lý này đảm bảo hệ thống vận hành chính xác, an ninh và phù hợp với
mô hình bãi đỗ xe thông minh trong thực tế.


\section{Mục tiêu và phạm vi của đề tài}

\subsection{Mục tiêu đề tài}

Mục tiêu của đề tài “Bãi đỗ xe thông minh tích hợp AI” là xây dựng một hệ thống quản lý bãi xe hiện đại, ứng dụng công nghệ nhận diện biển số, khuôn mặt và logo xe kết hợp với IoT và thanh toán điện tử nhằm tối ưu hóa quy trình ra vào bãi, nâng cao tính an toàn, minh bạch và tiện lợi cho người dùng. Hệ thống hướng tới việc tự động hóa kiểm soát xe máy và ô tô, giảm bớt sự phụ thuộc vào con người, đồng thời hỗ trợ bảo vệ và quản trị viên trong công tác giám sát, thống kê, xử lý sự cố. Ngoài ra, hệ thống còn tích hợp cổng thanh toán VNPAY giúp người dùng dễ dàng thanh toán phí giữ xe không dùng tiền mặt, đảm bảo tính nhanh chóng và an toàn.

\subsection{Phạm vi đề tài}

Phạm vi của đề tài tập trung triển khai cho hai loại phương tiện chính là xe máy và ô tô. Đối với xe máy, hệ thống quét biển số và nhận diện khuôn mặt người điều khiển khi ra vào. Đối với ô tô, khi vào sẽ nhận diện biển số, quét phần đuôi xe, chụp và lưu hình ảnh đầu xe, đồng thời nhận diện logo, khi ra sẽ tiếp tục quét biển số cả đầu và đuôi xe, sử dụng mạng Siamese để so sánh hình ảnh và logo xe nhằm xác thực. Hệ thống hỗ trợ quét mã QR do bảo vệ cấp trong các trường hợp đặc biệt. Ứng dụng mobile cho người dùng cung cấp chức năng đăng ký, đăng nhập, quản lý thông tin phương tiện, thao tác rời bãi và thanh toán phí giữ xe. Ứng dụng mobile cho bảo vệ được phép theo dõi lịch sử xe ra vào, tạo và gửi yêu cầu mở cổng cho quản trị viên. Trang web dành cho admin được xây dựng để quản lý, thống kê lượt xe ra vào và xử lý yêu cầu từ bảo vệ. Về phần IoT, hệ thống cổng ra vào được tự động hóa, hiển thị hình ảnh lên màn hình và phát cảnh báo khi phát hiện hành vi trái phép.


\chapter{KIỂM THỬ VÀ ĐÁNH GIÁ}
\label{chap:chap4-testing-and-evaluation}


\chapter*{\centering\Large{KẾT LUẬN VÀ PHÁT TRIỂN}}
\addcontentsline{toc}{chapter}{KẾT LUẬN VÀ PHÁT TRIỂN}

% Renumber sections to show only the section number (1, 2, 3...) instead of chapter.section (0.1, 0.2, 0.3...)
\renewcommand{\thesection}{\arabic{section}}
\setcounter{section}{0}

\section{Kết luận}

Đề tài “Bãi đỗ xe thông minh” đã xây dựng và triển khai thành công một hệ thống quản lý bãi xe hiện đại, ứng dụng các công nghệ như nhận diện hình ảnh, IoT và thanh toán điện tử. Hệ thống đáp ứng tốt các mục tiêu đề ra, bao gồm:

\begin{itemize}
    \item Tự động hóa quy trình quản lý ra/vào đối với xe máy và ô tô, giảm sự phụ thuộc vào con người.
    \item Đảm bảo an toàn, minh bạch trong giám sát và quản lý bãi xe.
    \item Nâng cao trải nghiệm người dùng thông qua ứng dụng di động và tích hợp thanh toán điện tử.
    \item Hỗ trợ quản trị tập trung với hệ thống web quản lý trực quan.
\end{itemize}

Kết quả kiểm thử cho thấy hệ thống có độ chính xác cao, hoạt động ổn định và có khả năng ứng dụng thực tế tại các bãi xe quy mô vừa và nhỏ. Mặc dù còn tồn tại một số hạn chế về điều kiện môi trường, kết nối mạng và chi phí triển khai, đề tài đã chứng minh được tính khả thi và tiềm năng ứng dụng trong thực tế, góp phần hướng tới phát triển giao thông và đô thị thông minh.

\section{Hướng phát triển trong tương lai}

Trong thời gian tới, hệ thống bãi đỗ xe thông minh có thể được tiếp tục phát triển theo các định hướng sau:

\subsection{Nâng cao độ chính xác và khả năng nhận diện}

Tiếp tục cải thiện các mô hình nhận diện nhằm tăng độ chính xác trong các điều kiện môi trường phức tạp như ánh sáng yếu, biển số bị che khuất hoặc phương tiện di chuyển nhanh, đồng thời bổ sung cơ chế tự học để hệ thống thích nghi tốt hơn với dữ liệu mới.

\subsection{Mở rộng phạm vi hỗ trợ và đối tượng sử dụng}

Mở rộng hệ thống để hỗ trợ thêm nhiều loại phương tiện và hình thức xác thực khác nhau, đáp ứng nhu cầu sử dụng đa dạng tại các khu vực công cộng, khu công nghiệp và khu đô thị.

\subsection{Tối ưu hóa hạ tầng và hiệu năng hệ thống}

Nâng cao hiệu năng xử lý, giảm độ trễ và tăng tính ổn định của hệ thống, đặc biệt trong các tình huống mất kết nối mạng hoặc lưu lượng phương tiện lớn.

\subsection{Mở rộng khả năng tích hợp và quy mô triển khai}

Phát triển hệ thống theo hướng dễ dàng tích hợp với các nền tảng quản lý khác và triển khai trên quy mô lớn, từ bãi xe đơn lẻ đến mô hình quản lý tập trung nhiều bãi xe.

\chapter*{\centering\Large{TÀI LIỆU THAM KHẢO}}
\addcontentsline{toc}{chapter}{TÀI LIỆU THAM KHẢO}

Alexander Mordvintsev \& Abid K. (2017). OpenCV-Python Tutorials seasion 1.

Python interface to the Google's Firebase REST APIs. (2023, 2 22). From pyqi: \url{https://pypi.org/project/firebase/}

Ultralytics. (2023, 1 10). Ultralytics. From modelYolov8: \url{https://docs.ultralytics.com/vi/models/yolov8/}




% 15. TÀI LIỆU THAM KHẢO
\clearpage

\begin{thebibliography}{99}
\addcontentsline{toc}{chapter}{TÀI LIỆU THAM KHẢO}

\bibitem{siamese_gfg}
GeeksforGeeks, "Siamese neural network in deep learning," GeeksforGeeks, 2024. [Online]. Available: https://www.geeksforgeeks.org/nlp/siamese-neural-network-in-deep-learning/. [Accessed: Dec. 20, 2025].

\bibitem{yolov8_ultralytics}
Ultralytics, "YOLOv8 models," Ultralytics Documentation, 2024. [Online]. Available: https://docs.ultralytics.com/vi/models/yolov8/. [Accessed: Dec. 20, 2025].

\bibitem{siamese_hackernoon}
"One-shot learning with siamese networks in PyTorch," HackerNoon, 2021. [Online]. Available: https://hackernoon.com/one-shot-learning-with-siamese-networks-in-pytorch-8ddaab10340e. [Accessed: Dec. 20, 2025].

\bibitem{cnn_linkedin}
LinkedIn, "How can you train a convolutional neural network," LinkedIn Learning, 2023. [Online]. Available: https://www.linkedin.com/advice/3/how-can-you-train-convolutional-neural-network-dtcye. [Accessed: Dec. 20, 2025].

\bibitem{deepface_viso}
Viso.ai, "DeepFace: Face recognition with deep learning," Viso.ai Computer Vision Blog, 2023. [Online]. Available: https://viso.ai/computer-vision/deepface/. [Accessed: Dec. 20, 2025].

\bibitem{jetpack_compose_mvvm}
T. GIS, "Creating lists in Android Jetpack Compose using MVVM with ViewModel in Kotlin," Medium, Aug. 2023. [Online]. Available: https://tomasgis.com/creating-lists-in-android-jetpack-compose-using-mvvm-with-viewmodel-in-kotlin-dcfc76945ae7. [Accessed: Dec. 20, 2025].



\end{thebibliography}
\end{document}